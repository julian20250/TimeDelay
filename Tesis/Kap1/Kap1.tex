\addcontentsline{toc}{chapter}{\numberline{}Introduction}
\chapter*{Introduction}
In 1916 Albert Einstein predicted the existence of gravitational radiation, see \cite{1916} and \cite{1918}, this under a weak field assumption also known as linearized gravity. This field then was vastly explored, even generalized to be able to work it with out a weak field assumption. Several of the ideas that were develop for the study of gravitational radiation were brought from the classical electrodynamics, but with the precaution of not confusing the concepts of electrodynamics and gravitation. Still the observational part was still missing.\\
The first observation of gravitational radiation was made by Hulse and Taylor in 1975 with the discovery of the pulsar PSR B1913$+$16, see \cite{HULSE}. It was such a discovery that in 1993 they earned the Nobel Price of physics, but this was an indirect observation. It wasn't until 2015 that LIGO and Virgo interferometers observed a transient gravitational-wave signal of the inspiral and merger of a pair of black holes, see \cite{LIGO}. With this finally was a direct observation of gravitational radiation, and it was not the only observation that these interferometers detected, see \cite{VIRGO}.\\
The purpose of this text is to introduce the basic mathematical and physical knowledge of gravitational radiation. For this, it has been divided in four chapters. Chapter one is an introduction to Differential Geometry and General Relativity, in the next chapter several mathematical concepts became really important. In chapter two is studied the relation between linearized gravity and gravitational radiation, this having in mind that this is only valid in a particular zone of radiation. The Einstein field equations are written as a wave equation in the zone of radiation where linearized gravity works. It is also study the gauge transformations, energy and multipole contribution. In Chapter 3 the Einstein field equations as a wave equation are shown again, but in this case with out any particular approximation over the gravitational filed, also expressions for the energy, lineal and angular momentum are obtained in general. Finally, in chapter four is shown the relation between gravitational radiation in linearized gravity and the Weyl tensor, but it is necessary to introduce first the tetrads, the Newman-Penrose formalism and the Weyl scalars. Once that this is introduced,  the flux energy, lineal and angular momentum expressions are written in terms of a Weyl scalar. Two appendix are given, one is for the equivalence of the harmonic coordinates and the DeDonder gauge, and the other one is a deeper study of the relativistic angular momentum.