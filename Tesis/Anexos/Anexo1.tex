%% LyX 2.2.3 created this file.  For more info, see http://www.lyx.org/.
%% Do not edit unless you really know what you are doing.
\documentclass[english]{article}
\usepackage[T1]{fontenc}
\usepackage[latin9]{inputenc}
\usepackage{geometry}
\geometry{verbose,tmargin=2cm,bmargin=2cm,lmargin=3cm,rmargin=3.5cm}
\usepackage{amsmath}
\usepackage{stmaryrd}
\usepackage{babel}
\begin{document}

\title{Appendix 1: Harmonic coordinates(Dirac)}

\maketitle
In this appendix we will proof some properties that are going to be
use to arrived, from the harmonic coordinates, to the DeDonder gauge.
Main references are DIRAC, LANDAU.

\section{Useful properties}

The following properties are going to be useful for this appendix

\subsection*{Property 1}

\begin{equation}
\partial_{\sigma}g^{\alpha\beta}=-g^{\alpha\mu}g^{\beta\nu}\partial_{\sigma}g_{\mu\nu},\label{eq:a1}
\end{equation}
\textbf{Proof:} We start from
\begin{align*}
\partial_{\sigma}g^{\alpha\mu}\cdot g_{\mu\nu}+g^{\alpha\mu}\cdot\partial_{\sigma}g_{\mu\nu} & =\partial_{\sigma}\left(g^{\alpha\mu}g_{\mu\nu}\right)\\
\  & =\partial_{\sigma}\delta_{\nu}^{\alpha}\\
\  & =0
\end{align*}
then
\[
\partial_{\sigma}g^{\alpha\mu}\cdot g_{\mu\nu}=-g^{\alpha\mu}\cdot\partial_{\sigma}g_{\mu\nu},
\]
multiplying by $g^{\beta\nu}$ and adding in $\nu$, because $g^{\beta\nu}g_{\mu\nu}=\delta_{\mu}^{\beta}$,
we obtain (\ref{eq:a1}).

\subsection*{Property 2}

\begin{equation}
\Gamma_{\alpha\beta\sigma}+\Gamma_{\beta\alpha\sigma}=\partial_{\sigma}g_{\alpha\beta},\label{eq:a2}
\end{equation}
\textbf{Proof:} From the fact that
\[
\Gamma_{\beta\gamma}^{\alpha}=\frac{1}{2}g^{\alpha\sigma}\left(\partial_{\beta}g_{\sigma\gamma}+\partial_{\gamma}g_{\sigma\beta}-\partial_{\sigma}g_{\beta\gamma}\right)
\]
we have that
\[
\Gamma_{\alpha\beta\sigma}=\frac{1}{2}\left(\partial_{\beta}g_{\alpha\sigma}+\partial_{\sigma}g_{\alpha\beta}-\partial_{\alpha}g_{\beta\sigma}\right)
\]
and

\[
\Gamma_{\beta\alpha\sigma}=\frac{1}{2}\left(\partial_{\alpha}g_{\beta\sigma}+\partial_{\sigma}g_{\beta\alpha}-\partial_{\beta}g_{\alpha\sigma}\right),
\]
if we make $\Gamma_{\alpha\beta\sigma}+\Gamma_{\beta\alpha\sigma}$
we obtain (\ref{eq:a2}).

\subsection*{Property 3}

\begin{equation}
\Gamma_{\alpha\beta}^{\beta}\sqrt{-g}=\partial_{\text{\ensuremath{\alpha}}}\sqrt{-g},\label{eq:a3}
\end{equation}
\textbf{Proof: }First we have to differentiate the determinant of
the metric tensor $g$, we must differentiate each element $g_{\lambda\mu}$
in it and then multiply by the cofactor $gg^{\lambda u}$. Thus
\begin{equation}
\partial_{\nu}g=gg^{\lambda\mu}\partial_{\nu}g_{\lambda\mu}.\label{eq:ag}
\end{equation}
Now we are going to calculate $\Gamma_{\nu\mu}^{\mu}$
\begin{align}
\Gamma_{\nu\mu}^{\mu} & =\frac{1}{2}g^{\mu\sigma}\left(\partial_{\mu}g_{\sigma\nu}+\partial_{\nu}g_{\sigma\mu}-\partial_{\sigma}g_{\nu\mu}\right)\\
\  & =\frac{1}{2}\left(g^{\mu\sigma}\partial_{\mu}g_{\sigma\nu}-g^{\mu\sigma}\partial_{\sigma}g_{\nu\mu}+g^{\mu\sigma}\partial_{\nu}g_{\sigma\mu}\right)\\
\  & =\frac{1}{2}g^{\mu\sigma}\partial_{\nu}g_{\sigma\mu}.\label{eq:a-gamma}
\end{align}
Let us write $g^{-1}\partial_{\nu}g$ in the following way
\begin{align}
g^{-1}\partial_{\nu}g & =g^{-1}\partial_{\nu}\left(\left[\sqrt{-g}\right]^{2}\right)\nonumber \\
\  & =2\left(\sqrt{-g}\right)^{-1}\partial_{\nu}\sqrt{-g},\label{eq:a-sqrt}
\end{align}
from (\ref{eq:ag}) and (\ref{eq:a-sqrt})
\begin{equation}
\partial_{\nu}\sqrt{-g}=\frac{1}{2}\sqrt{-g}g^{\lambda\mu}\partial_{\nu}g_{\lambda\mu}.\label{eq:a-gg}
\end{equation}
Using (\ref{eq:a-gg}) in (\ref{eq:a-gamma}) we obtain (\ref{eq:a3}).

\section{The DeDonder Gauge}

To understand harmonic coordinates we have to start from the d'Alembert
equation for a scalar $V$, namely $\boxempty V=0$, in a curved spacetime
\[
\boxempty V=g^{\alpha\beta}\nabla_{\alpha}\nabla_{\beta}V,
\]
then
\[
g^{\alpha\beta}\nabla_{\alpha}\nabla_{\beta}V=g^{\alpha\beta}\nabla_{\alpha}\left(\partial_{\beta}V\right)=g^{\alpha\beta}\left(\partial_{\alpha}\partial_{\beta}V-\Gamma_{\alpha\beta}^{\sigma}\partial_{\sigma}V\right)=0,
\]
if we are using rectilinear axes in flat space, each of the four coordinates
$x^{\alpha}$ satisfies $\boxempty x^{\alpha}=0$. This impose a restriction
over the coordinates, because $x^{\alpha}$ is not an scalar like
$V$, so it holds only in certain coordinate system.

If we substitute $x^{\alpha}$ for $V$
\begin{align*}
g^{\alpha\beta}\left(\partial_{\alpha}\partial_{\beta}x^{\lambda}-\Gamma_{\alpha\beta}^{\sigma}\partial_{\sigma}x^{\lambda}\right)= & \ g^{\alpha\beta}\left(\partial_{\alpha}\delta_{\text{\ensuremath{\beta}}}^{\lambda}-\Gamma_{\alpha\beta}^{\sigma}\delta_{\text{\ensuremath{\sigma}}}^{\lambda}\right)\\
= & \ -g^{\alpha\beta}\Gamma_{\alpha\beta}^{\lambda}=0,
\end{align*}
the coordinates that satisfies this condition are called \textbf{harmonic
coordinates}. They provide the closest approximation to rectilinear
coordinates that we can have in curved spacetime. Now, we are going
to prove that the harmonic coordinate system is equivalent to set
the DeDonder gauge $\partial_{\beta}\left(g^{\alpha\beta}\sqrt{-g}\right)=0$,
from equations (\ref{eq:a1}) and (\ref{eq:a2})

\begin{align*}
\partial_{\sigma}g^{\alpha\beta}= & \ -g^{\alpha\mu}g^{\beta\nu}\left(\Gamma_{\mu\nu\sigma}+\Gamma_{\nu\mu\sigma}\right)\\
= & \ -g^{\alpha\mu}\Gamma_{\mu\sigma}^{\beta}-g^{\beta\nu}\Gamma_{\nu\sigma}^{\alpha},
\end{align*}
from equation (\ref{eq:a2}) and (\ref{eq:a3}) 
\begin{align*}
\partial_{\sigma}\left(g^{\alpha\beta}\sqrt{-g}\right)= & \ \sqrt{-g}\partial_{\sigma}g^{\alpha\beta}+g^{\alpha\beta}\partial_{\sigma}\sqrt{-g}\\
= & \ \sqrt{-g}\left(-g^{\alpha\mu}\Gamma_{\mu\sigma}^{\beta}-g^{\beta\nu}\Gamma_{\nu\sigma}^{\alpha}+g^{\alpha\beta}\Gamma_{\sigma\mu}^{\mu}\right),
\end{align*}
contracting $\sigma$ and $\beta$
\begin{align*}
\partial_{\beta}\left(g^{\alpha\beta}\sqrt{-g}\right)= & \ \sqrt{-g}\left(-g^{\alpha\mu}\Gamma_{\mu\beta}^{\beta}-g^{\beta\nu}\Gamma_{\nu\beta}^{\alpha}+g^{\alpha\beta}\Gamma_{\beta\mu}^{\mu}\right)\\
= & \ -\sqrt{-g}g^{\beta\nu}\Gamma_{\nu\beta}^{\alpha},
\end{align*}
because $g^{\alpha\beta}\Gamma_{\alpha\beta}^{\lambda}=0$, then
\[
\partial_{\beta}\left(g^{\alpha\beta}\sqrt{-g}\right)=0
\]
which is what we wanted to proof.
\end{document}
