%% LyX 2.2.3 created this file.  For more info, see http://www.lyx.org/.
%% Do not edit unless you really know what you are doing.
\documentclass[english]{article}
\usepackage[T1]{fontenc}
\usepackage[latin9]{inputenc}
\usepackage{geometry}
\geometry{verbose,tmargin=2cm,bmargin=2cm,lmargin=3cm,rmargin=3.5cm}
\usepackage{amsmath}
\usepackage{babel}
\begin{document}

\title{Appendix 2: Relativistic Angular Momentum}
\maketitle

\section{For a closed particle system}

Here we are going to obtain a relativistic expression for the angular
momentum. We need to take into account that there is conservation
of angular momentum in classical mechanics, so here we are goint to
take a rotation and proof that there is an expression that is conserved
under rotations in the four-dimensional space and this expression
is the angular momentum. We also verify that this correspond to the
angular momentum expression when we return to a classical system. 

Let us take a system of several particles, let $x^{\alpha}$ be the
coordinates of one of the particles of the system. We make an infinitesimal
rotation in the four-dimensional space. Under a transformation, the
coordinates $x^{\alpha}$ take on new values $x'^{\alpha}$ such that
the differences $x'^{\alpha}-x^{\alpha}$ are linear functions
\begin{equation}
x'^{\alpha}-x^{\alpha}=x_{\beta}\delta\Omega^{\alpha\beta},\label{eq:b1}
\end{equation}
where $\delta\Omega^{\alpha\beta}$ are infinitesimal coefficients.
Under rotation, the length of the four-position vector must remain
unchaged
\begin{equation}
x'_{\alpha}x'^{\alpha}=x_{\alpha}x^{\alpha},\label{eq:b2}
\end{equation}
from equation (\ref{eq:b1}) $x'^{\alpha}=x^{\alpha}+x_{\beta}\delta\Omega^{\alpha\beta}$,
then
\begin{align*}
x'_{\alpha}x'^{\alpha}= & \ \left(x_{\alpha}+x_{\beta}\delta\Omega_{\alpha}^{\beta}\right)\left(x^{\alpha}+x_{\beta}\delta\Omega^{\alpha\beta}\right)\\
= & \ x_{\alpha}x^{\alpha}+x_{\alpha}x_{\beta}\delta\Omega^{\alpha\beta}+x^{\alpha}x_{\beta}\delta\Omega_{\alpha}^{\beta}+\mathcal{O}\left(\delta\Omega^{2}\right),
\end{align*}
because
\[
x_{\alpha}x_{\beta}\delta\Omega^{\alpha\beta}=x^{\alpha}x_{\beta}\delta\Omega_{\alpha}^{\beta}=x^{\alpha}x^{\beta}\delta\Omega_{\alpha\beta}
\]
we have that
\[
x'_{\alpha}x'^{\alpha}=x_{\alpha}x^{\alpha}+2x^{\alpha}x^{\beta}\delta\Omega_{\alpha\beta}+\mathcal{O}\left(\delta\Omega^{2}\right),
\]
taking only the first orden contribution $\mathcal{O}\left(\delta\Omega^{2}\right)=0$.
From equation (\ref{eq:b2})
\[
x^{\alpha}x^{\beta}\delta\Omega_{\alpha\beta}=0,
\]
since $x^{\alpha}x^{\beta}$ is a symmetric tensor $\delta\Omega_{\alpha\beta}$
must be antisymmetric, $\delta\Omega_{\alpha\beta}=-\delta\Omega_{\beta\alpha}$.

Now, we have that the action for a free material point is, (see LANDAU)
\[
S=-mc\int_{a}^{b}ds
\]
then $\delta S=-mc\int_{a}^{b}ds=0$, it can be shown that, (see LANDAU)
\[
\delta S=-p^{\alpha}\delta x_{\alpha},
\]
where $p^{\alpha}$ is the four-momentum. Adding over all particles
in the system
\[
\delta S=-\sum\left.p^{\alpha}\delta x_{\alpha}\right|_{a}^{b},
\]
which goes from the event $a$ to the event $b$, in the case of a
rotation $\delta x_{\alpha}=\delta\Omega_{\alpha\beta}x^{\beta}$,
therefore

\[
\delta S=-\delta\Omega_{\alpha\beta}\sum\left.p^{\alpha}x^{\beta}\right|_{a}^{b}.
\]
We split $\sum\left.p^{\alpha}x^{\beta}\right|_{a}^{b}$ into its
symmetric and antisymmetric part
\begin{align*}
\delta S= & \ -\delta\Omega_{\alpha\beta}\left[\left(\sum\left.p^{\alpha}x^{\beta}\right|_{a}^{b}\right)_{\text{symmetric}}+\left(\sum\left.p^{\alpha}x^{\beta}\right|_{a}^{b}\right)_{\text{antisymmetric}}\right]\\
= & \ -\delta\Omega_{\alpha\beta}\left(\sum\left.p^{\alpha}x^{\beta}\right|_{a}^{b}\right)_{\text{antisymmetric}},
\end{align*}
then
\[
\delta\Omega_{\alpha\beta}\left[\frac{1}{2}\sum\left.\left(p^{\alpha}x^{\beta}-p^{\beta}x^{\alpha}\right)\right|_{a}^{b}\right]=0.
\]
For a closed system, the action is not changed by a rotation in the
four-space, this means that the coefficients $\delta\Omega_{\alpha\beta}$
must vanish, according to this
\[
\sum\left.\left(p^{\alpha}x^{\beta}-p^{\beta}x^{\alpha}\right)\right|_{a}=\sum\left.\left(p^{\alpha}x^{\beta}-p^{\beta}x^{\alpha}\right)\right|_{b}.
\]
Consequently, for a closed system the antisymmetric tensor of angular
momentum $J^{\alpha\beta}$ is defined as
\[
J^{\alpha\beta}=\sum\left(p^{\alpha}x^{\beta}-p^{\beta}x^{\alpha}\right).
\]
We check that the spatial components of $J^{\alpha\beta}$ correspond
to the classical angular momentum
\[
J^{23}=J_{x},\ J^{31}=J_{y},\ J^{12}=J_{z}.
\]

\section{For a general system}

Here we consider a general system whose action integral has the form
\[
S=\int\Lambda\left(q,\partial_{\alpha}q\right)dVdt=\frac{1}{c}\int\Lambda\left(q,\partial_{\alpha}q\right)dx^{4},
\]
where $\Lambda$ is some function of the quantities $q$, describing
the state of the system, and of their first derivatives with respect
to coordinates and time, we also have that
\[
\int\Lambda dV
\]
is the lagrangian of the system and $\Lambda$ its lagrangian density.
The equations of motion for the lagrangian density are given by
\begin{equation}
\frac{\partial}{\partial x^{\alpha}}\left(\frac{\partial\Lambda}{\partial\left(\partial_{\alpha}q\right)}\right)-\frac{\partial\Lambda}{\partial q}=0,\label{eq:b3}
\end{equation}
we are going to use the fact that
\begin{equation}
\frac{\partial\Lambda}{\partial x^{\alpha}}=\frac{\partial\Lambda}{\partial q}\frac{\partial q}{\partial x^{\alpha}}+\frac{\partial\Lambda}{\partial\left(\partial_{\beta}q\right)}\frac{\partial\left(\partial_{\beta}q\right)}{\partial x^{\alpha}},\label{eq:b4}
\end{equation}
if we substitute the term $\partial\Lambda/\partial q$ given by equation
(\ref{eq:b3}) in (\ref{eq:b4})
\begin{equation}
\frac{\partial\Lambda}{\partial x^{\alpha}}=\frac{\partial}{\partial x^{\beta}}\left(\frac{\partial\Lambda}{\partial\left(\partial_{\beta}q\right)}\right)\partial_{\alpha}q+\frac{\partial\Lambda}{\partial\left(\partial_{\beta}q\right)}\frac{\partial\left(\partial_{\alpha}q\right)}{\partial x^{\beta}}=\frac{\partial}{\partial x^{\beta}}\left(\partial_{\alpha}q\frac{\partial\Lambda}{\partial\left(\partial_{\beta}q\right)}\right).\label{eq:b5}
\end{equation}

On the other hand, we can write
\[
\frac{\partial\Lambda}{\partial x^{\alpha}}=\delta_{\alpha}^{\beta}\frac{\partial\Lambda}{\partial x^{\beta}},
\]
then we write equation (\ref{eq:b5}) as
\[
\frac{\partial}{\partial x^{\beta}}\left(\partial_{\alpha}q\frac{\partial\Lambda}{\partial\left(\partial_{\beta}q\right)}-\delta_{\alpha}^{\beta}\Lambda\right)=0,
\]
let us introduce the tensor $T_{\alpha}^{\beta}$ definde as
\[
T_{\alpha}^{\beta}=\partial_{\alpha}q\frac{\partial\Lambda}{\partial\left(\partial_{\beta}q\right)}-\delta_{\alpha}^{\beta}\Lambda,
\]
then
\[
\frac{\partial T_{\alpha}^{\beta}}{\partial x^{\beta}}=0
\]
this is satisfied for several quantities $q^{(l)}$, then

\[
T_{\alpha}^{\beta}=\sum_{l}\partial_{\alpha}q^{(l)}\frac{\partial\Lambda}{\partial\left(\partial_{\beta}q^{(l)}\right)}-\delta_{\alpha}^{\beta}\Lambda.
\]

We have the four divergence of a vector equal to zero, then the integral
over a hypersurface which contains all of three-dimensional space
is conserved. Given the units of the lagrangian density $T^{00}$
must be considered the energy density of the system, then $\int T^{00}dV$
is the total energy o the system. We can multiply our integral by
a constant and the integral is still conserved, we set this constant
to $1/c$, so $P^{0}$ is equal to the energy of the system multiplied
by $1/c$, therefore we get four momentum of the system expression
\[
P^{\alpha}=\frac{1}{c}\int T^{\alpha\beta}dS_{\beta},
\]
which agrees with the expression obtained solving the Killing equation,
see DEWITT. 

From the aboves section we have that we can define the angular momentum
as
\[
J^{\alpha\beta}=\frac{1}{c}\int\left(x^{\alpha}T^{\beta\sigma}-x^{\alpha}T^{\beta\sigma}\right)dS_{\sigma},
\]
lets see that the law of conservation of angular momentum implies
symmetric index in $T^{\alpha\beta}$, this law of conservation is
given by
\[
\partial_{\sigma}\left(x^{\alpha}T^{\beta\sigma}-x^{\alpha}T^{\beta\sigma}\right)=0,
\]
then
\begin{align*}
\delta_{\sigma}^{\alpha}T^{\beta\sigma}+x^{\alpha}\partial_{\sigma}T^{\beta\sigma}-\delta_{\sigma}^{\beta}T^{\alpha\sigma}-x^{\beta}\partial_{\sigma}T^{\alpha\sigma}= & \ \delta_{\sigma}^{\alpha}T^{\beta\sigma}-\delta_{\sigma}^{\beta}T^{\alpha\sigma}\\
= & \ T^{\alpha\beta}-T^{\beta\alpha}\\
= & 0,
\end{align*}
therefore $T^{\alpha\beta}=T^{\beta\alpha}$.
\end{document}
