\begin{appendix}
\chapter{Appendix A: Harmonic coordinates}\label{appendix-harm}
In this appendix first we will proof some properties of the Christoffel symbols and the metric tensor components. Then these properties are going to be
use to arrived, from the harmonic coordinates, to the DeDonder gauge.
Main reference for this appendix is \cite{DIRAC}.

\section{Useful properties}

The following properties are going to be useful for this appendix

\subsection*{Property 1}

\begin{equation}
\partial_{\sigma}g^{\alpha\beta}=-g^{\alpha\mu}g^{\beta\nu}\partial_{\sigma}g_{\mu\nu},\label{eq:a1}
\end{equation}
\textbf{Proof:} We start from
\begin{align*}
\partial_{\sigma}g^{\alpha\mu}\cdot g_{\mu\nu}+g^{\alpha\mu}\cdot\partial_{\sigma}g_{\mu\nu} & =\partial_{\sigma}\left(g^{\alpha\mu}g_{\mu\nu}\right)\\
\  & =\partial_{\sigma}\delta_{\nu}^{\alpha}\\
\  & =0
\end{align*}
then
\[
\partial_{\sigma}g^{\alpha\mu}\cdot g_{\mu\nu}=-g^{\alpha\mu}\cdot\partial_{\sigma}g_{\mu\nu},
\]
multiplying by $g^{\beta\nu}$ and adding in $\nu$, because $g^{\beta\nu}g_{\mu\nu}=\delta_{\mu}^{\beta}$,
we obtain (\ref{eq:a1}).

\subsection*{Property 2}

\begin{equation}
\Gamma_{\alpha\beta\sigma}+\Gamma_{\beta\alpha\sigma}=\partial_{\sigma}g_{\alpha\beta},\label{eq:a2}
\end{equation}
\textbf{Proof:} From the fact that
\[
\Gamma_{\beta\gamma}^{\alpha}=\frac{1}{2}g^{\alpha\sigma}\left(\partial_{\beta}g_{\sigma\gamma}+\partial_{\gamma}g_{\sigma\beta}-\partial_{\sigma}g_{\beta\gamma}\right)
\]
we have that
\[
\Gamma_{\alpha\beta\sigma}=\frac{1}{2}\left(\partial_{\beta}g_{\alpha\sigma}+\partial_{\sigma}g_{\alpha\beta}-\partial_{\alpha}g_{\beta\sigma}\right)
\]
and

\[
\Gamma_{\beta\alpha\sigma}=\frac{1}{2}\left(\partial_{\alpha}g_{\beta\sigma}+\partial_{\sigma}g_{\beta\alpha}-\partial_{\beta}g_{\alpha\sigma}\right),
\]
if we make $\Gamma_{\alpha\beta\sigma}+\Gamma_{\beta\alpha\sigma}$
we obtain (\ref{eq:a2}).

\subsection*{Property 3}

\begin{equation}
\Gamma_{\alpha\beta}^{\beta}\sqrt{-g}=\partial_{\text{\ensuremath{\alpha}}}\sqrt{-g},\label{eq:a3}
\end{equation}
\textbf{Proof: }First we have to differentiate the determinant of
the metric tensor $g$, we must differentiate each element $g_{\lambda\mu}$
in it and then multiply by the cofactor $gg^{\lambda u}$. Thus
\begin{equation}
\partial_{\nu}g=gg^{\lambda\mu}\partial_{\nu}g_{\lambda\mu}.\label{eq:ag}
\end{equation}
Now we are going to calculate $\Gamma_{\nu\mu}^{\mu}$
\begin{align}
\Gamma_{\nu\mu}^{\mu} & =\frac{1}{2}g^{\mu\sigma}\left(\partial_{\mu}g_{\sigma\nu}+\partial_{\nu}g_{\sigma\mu}-\partial_{\sigma}g_{\nu\mu}\right)\\
\  & =\frac{1}{2}\left(g^{\mu\sigma}\partial_{\mu}g_{\sigma\nu}-g^{\mu\sigma}\partial_{\sigma}g_{\nu\mu}+g^{\mu\sigma}\partial_{\nu}g_{\sigma\mu}\right)\\
\  & =\frac{1}{2}g^{\mu\sigma}\partial_{\nu}g_{\sigma\mu}.\label{eq:a-gamma}
\end{align}
Let us write $g^{-1}\partial_{\nu}g$ in the following way
\begin{align}
g^{-1}\partial_{\nu}g & =g^{-1}\partial_{\nu}\left(\left[\sqrt{-g}\right]^{2}\right)\nonumber \\
\  & =2\left(\sqrt{-g}\right)^{-1}\partial_{\nu}\sqrt{-g},\label{eq:a-sqrt}
\end{align}
from (\ref{eq:ag}) and (\ref{eq:a-sqrt})
\begin{equation}
\partial_{\nu}\sqrt{-g}=\frac{1}{2}\sqrt{-g}g^{\lambda\mu}\partial_{\nu}g_{\lambda\mu}.\label{eq:a-gg}
\end{equation}
Using (\ref{eq:a-gg}) in (\ref{eq:a-gamma}) we obtain (\ref{eq:a3}).

\section{The DeDonder Gauge}

To understand harmonic coordinates we have to start from the d'Alembert
equation for a scalar $V$, namely $\boxempty V=0$, in a curved spacetime
\[
\boxempty V=g^{\alpha\beta}\nabla_{\alpha}\nabla_{\beta}V,
\]
then
\[
g^{\alpha\beta}\nabla_{\alpha}\nabla_{\beta}V=g^{\alpha\beta}\nabla_{\alpha}\left(\partial_{\beta}V\right)=g^{\alpha\beta}\left(\partial_{\alpha}\partial_{\beta}V-\Gamma_{\alpha\beta}^{\sigma}\partial_{\sigma}V\right)=0,
\]
if we are using rectilinear axes in flat space, each of the four coordinates
$x^{\alpha}$ satisfies $\boxempty x^{\alpha}=0$. This impose a restriction
over the coordinates, because $x^{\alpha}$ is not an scalar like
$V$, so it holds only in certain coordinate system.

If we substitute $x^{\alpha}$ for $V$
\begin{align*}
g^{\alpha\beta}\left(\partial_{\alpha}\partial_{\beta}x^{\lambda}-\Gamma_{\alpha\beta}^{\sigma}\partial_{\sigma}x^{\lambda}\right)= & \ g^{\alpha\beta}\left(\partial_{\alpha}\delta_{\text{\ensuremath{\beta}}}^{\lambda}-\Gamma_{\alpha\beta}^{\sigma}\delta_{\text{\ensuremath{\sigma}}}^{\lambda}\right)\\
= & \ -g^{\alpha\beta}\Gamma_{\alpha\beta}^{\lambda}=0,
\end{align*}
the coordinates that satisfies this condition are called \textbf{harmonic
coordinates}. They provide the closest approximation to rectilinear
coordinates that we can have in curved spacetime. Now, we are going
to prove that the harmonic coordinate system is equivalent to set
the DeDonder gauge $\partial_{\beta}\left(g^{\alpha\beta}\sqrt{-g}\right)=0$,
from equations (\ref{eq:a1}) and (\ref{eq:a2})

\begin{align*}
\partial_{\sigma}g^{\alpha\beta}= & \ -g^{\alpha\mu}g^{\beta\nu}\left(\Gamma_{\mu\nu\sigma}+\Gamma_{\nu\mu\sigma}\right)\\
= & \ -g^{\alpha\mu}\Gamma_{\mu\sigma}^{\beta}-g^{\beta\nu}\Gamma_{\nu\sigma}^{\alpha},
\end{align*}
from equation (\ref{eq:a2}) and (\ref{eq:a3}) 
\begin{align*}
\partial_{\sigma}\left(g^{\alpha\beta}\sqrt{-g}\right)= & \ \sqrt{-g}\partial_{\sigma}g^{\alpha\beta}+g^{\alpha\beta}\partial_{\sigma}\sqrt{-g}\\
= & \ \sqrt{-g}\left(-g^{\alpha\mu}\Gamma_{\mu\sigma}^{\beta}-g^{\beta\nu}\Gamma_{\nu\sigma}^{\alpha}+g^{\alpha\beta}\Gamma_{\sigma\mu}^{\mu}\right),
\end{align*}
contracting $\sigma$ and $\beta$
\begin{align*}
\partial_{\beta}\left(g^{\alpha\beta}\sqrt{-g}\right)= & \ \sqrt{-g}\left(-g^{\alpha\mu}\Gamma_{\mu\beta}^{\beta}-g^{\beta\nu}\Gamma_{\nu\beta}^{\alpha}+g^{\alpha\beta}\Gamma_{\beta\mu}^{\mu}\right)\\
= & \ -\sqrt{-g}g^{\beta\nu}\Gamma_{\nu\beta}^{\alpha},
\end{align*}
because $g^{\alpha\beta}\Gamma_{\alpha\beta}^{\lambda}=0$, then
\[
\partial_{\beta}\left(g^{\alpha\beta}\sqrt{-g}\right)=0
\]
which is what we wanted to proof.
%%%%%%%%%%%%%%%%%%%%%%%%%%%%%%%%%%%%%%%%%%%%%%%%%%%%%%%%%%%
\chapter{Appendix B: Relativistic Angular Momentum}\label{appendix-momentum}
In this appendix will be obtain the expression for the relativistic angular momentum for a closed particle system. Then a general system is consider with a general action integral, a momentum and angular momentum expression is obtained and the role of the angular momentum conservation in the symmetry of the energy-momentum tensor. Main reference is \cite{LANDAU,DEWITT}.

\section{For a closed particle system}

Here we are going to obtain a relativistic expression for the angular
momentum. We need to take into account that there is conservation
of angular momentum in classical mechanics, so here we are goint to
take a rotation and proof that there is an expression that is conserved
under rotations in the four-dimensional space and this expression
is the angular momentum. We also verify that this correspond to the
angular momentum expression when we return to a classical system. 

Let us take a system of several particles, let $x^{\alpha}$ be the
coordinates of one of the particles of the system. We make an infinitesimal
rotation in the four-dimensional space. Under a transformation, the
coordinates $x^{\alpha}$ take on new values $x'^{\alpha}$ such that
the differences $x'^{\alpha}-x^{\alpha}$ are linear functions
\begin{equation}
x'^{\alpha}-x^{\alpha}=x_{\beta}\delta\Omega^{\alpha\beta},\label{eq:b1}
\end{equation}
where $\delta\Omega^{\alpha\beta}$ are infinitesimal coefficients.
Under rotation, the length of the four-position vector must remain
unchaged
\begin{equation}
x'_{\alpha}x'^{\alpha}=x_{\alpha}x^{\alpha},\label{eq:b2}
\end{equation}
from equation (\ref{eq:b1}) $x'^{\alpha}=x^{\alpha}+x_{\beta}\delta\Omega^{\alpha\beta}$,
then
\begin{align*}
x'_{\alpha}x'^{\alpha}= & \ \left(x_{\alpha}+x_{\beta}\delta\Omega_{\alpha}^{\beta}\right)\left(x^{\alpha}+x_{\beta}\delta\Omega^{\alpha\beta}\right)\\
= & \ x_{\alpha}x^{\alpha}+x_{\alpha}x_{\beta}\delta\Omega^{\alpha\beta}+x^{\alpha}x_{\beta}\delta\Omega_{\alpha}^{\beta}+\mathcal{O}\left(\delta\Omega^{2}\right),
\end{align*}
because
\[
x_{\alpha}x_{\beta}\delta\Omega^{\alpha\beta}=x^{\alpha}x_{\beta}\delta\Omega_{\alpha}^{\beta}=x^{\alpha}x^{\beta}\delta\Omega_{\alpha\beta}
\]
we have that
\[
x'_{\alpha}x'^{\alpha}=x_{\alpha}x^{\alpha}+2x^{\alpha}x^{\beta}\delta\Omega_{\alpha\beta}+\mathcal{O}\left(\delta\Omega^{2}\right),
\]
taking only the first orden contribution $\mathcal{O}\left(\delta\Omega^{2}\right)=0$.
From equation (\ref{eq:b2})
\[
x^{\alpha}x^{\beta}\delta\Omega_{\alpha\beta}=0,
\]
since $x^{\alpha}x^{\beta}$ is a symmetric tensor $\delta\Omega_{\alpha\beta}$
must be antisymmetric, $\delta\Omega_{\alpha\beta}=-\delta\Omega_{\beta\alpha}$.

Now, we have that the action for a free material point is, (see \cite{LANDAU})
\[
S=-mc\int_{a}^{b}ds
\]
then $\delta S=-mc\int_{a}^{b}ds=0$, it can be shown that, (see \cite{LANDAU})
\[
\delta S=-p^{\alpha}\delta x_{\alpha},
\]
where $p^{\alpha}$ is the four-momentum. Adding over all particles
in the system
\[
\delta S=-\sum\left.p^{\alpha}\delta x_{\alpha}\right|_{a}^{b},
\]
which goes from the event $a$ to the event $b$, in the case of a
rotation $\delta x_{\alpha}=\delta\Omega_{\alpha\beta}x^{\beta}$,
therefore

\[
\delta S=-\delta\Omega_{\alpha\beta}\sum\left.p^{\alpha}x^{\beta}\right|_{a}^{b}.
\]
We split $\sum\left.p^{\alpha}x^{\beta}\right|_{a}^{b}$ into its
symmetric and antisymmetric part
\begin{align*}
\delta S= & \ -\delta\Omega_{\alpha\beta}\left[\left(\sum\left.p^{\alpha}x^{\beta}\right|_{a}^{b}\right)_{\text{symmetric}}+\left(\sum\left.p^{\alpha}x^{\beta}\right|_{a}^{b}\right)_{\text{antisymmetric}}\right]\\
= & \ -\delta\Omega_{\alpha\beta}\left(\sum\left.p^{\alpha}x^{\beta}\right|_{a}^{b}\right)_{\text{antisymmetric}},
\end{align*}
then
\[
\delta\Omega_{\alpha\beta}\left[\frac{1}{2}\sum\left.\left(p^{\alpha}x^{\beta}-p^{\beta}x^{\alpha}\right)\right|_{a}^{b}\right]=0.
\]
For a closed system, the action is not changed by a rotation in the
four-space, this means that the coefficients $\delta\Omega_{\alpha\beta}$
must vanish, according to this
\[
\sum\left.\left(p^{\alpha}x^{\beta}-p^{\beta}x^{\alpha}\right)\right|_{a}=\sum\left.\left(p^{\alpha}x^{\beta}-p^{\beta}x^{\alpha}\right)\right|_{b}.
\]
Consequently, for a closed system the antisymmetric tensor of angular
momentum $J^{\alpha\beta}$ is defined as
\[
J^{\alpha\beta}=\sum\left(p^{\alpha}x^{\beta}-p^{\beta}x^{\alpha}\right).
\]
We check that the spatial components of $J^{\alpha\beta}$ correspond
to the classical angular momentum
\[
J^{23}=J_{x},\ J^{31}=J_{y},\ J^{12}=J_{z}.
\]

\section{For a general system}

Here we consider a general system whose action integral has the form
\[
S=\int\Lambda\left(q,\partial_{\alpha}q\right)dVdt=\frac{1}{c}\int\Lambda\left(q,\partial_{\alpha}q\right)dx^{4},
\]
where $\Lambda$ is some function of the quantities $q$, describing
the state of the system, and of their first derivatives with respect
to coordinates and time, we also have that
\[
\int\Lambda dV
\]
is the lagrangian of the system and $\Lambda$ its lagrangian density.
The equations of motion for the lagrangian density are given by
\begin{equation}
\frac{\partial}{\partial x^{\alpha}}\left(\frac{\partial\Lambda}{\partial\left(\partial_{\alpha}q\right)}\right)-\frac{\partial\Lambda}{\partial q}=0,\label{eq:b3}
\end{equation}
we are going to use the fact that
\begin{equation}
\frac{\partial\Lambda}{\partial x^{\alpha}}=\frac{\partial\Lambda}{\partial q}\frac{\partial q}{\partial x^{\alpha}}+\frac{\partial\Lambda}{\partial\left(\partial_{\beta}q\right)}\frac{\partial\left(\partial_{\beta}q\right)}{\partial x^{\alpha}},\label{eq:b4}
\end{equation}
if we substitute the term $\partial\Lambda/\partial q$ given by equation
(\ref{eq:b3}) in (\ref{eq:b4})
\begin{equation}
\frac{\partial\Lambda}{\partial x^{\alpha}}=\frac{\partial}{\partial x^{\beta}}\left(\frac{\partial\Lambda}{\partial\left(\partial_{\beta}q\right)}\right)\partial_{\alpha}q+\frac{\partial\Lambda}{\partial\left(\partial_{\beta}q\right)}\frac{\partial\left(\partial_{\alpha}q\right)}{\partial x^{\beta}}=\frac{\partial}{\partial x^{\beta}}\left(\partial_{\alpha}q\frac{\partial\Lambda}{\partial\left(\partial_{\beta}q\right)}\right).\label{eq:b5}
\end{equation}

On the other hand, we can write
\[
\frac{\partial\Lambda}{\partial x^{\alpha}}=\delta_{\alpha}^{\beta}\frac{\partial\Lambda}{\partial x^{\beta}},
\]
then we write equation (\ref{eq:b5}) as
\[
\frac{\partial}{\partial x^{\beta}}\left(\partial_{\alpha}q\frac{\partial\Lambda}{\partial\left(\partial_{\beta}q\right)}-\delta_{\alpha}^{\beta}\Lambda\right)=0,
\]
let us introduce the tensor $T_{\alpha}^{\beta}$ definde as
\[
T_{\alpha}^{\beta}=\partial_{\alpha}q\frac{\partial\Lambda}{\partial\left(\partial_{\beta}q\right)}-\delta_{\alpha}^{\beta}\Lambda,
\]
then
\[
\frac{\partial T_{\alpha}^{\beta}}{\partial x^{\beta}}=0
\]
this is satisfied for several quantities $q^{(l)}$, then

\[
T_{\alpha}^{\beta}=\sum_{l}\partial_{\alpha}q^{(l)}\frac{\partial\Lambda}{\partial\left(\partial_{\beta}q^{(l)}\right)}-\delta_{\alpha}^{\beta}\Lambda.
\]

We have the four divergence of a vector equal to zero, then the integral
over a hypersurface which contains all of three-dimensional space
is conserved. Given the units of the lagrangian density $T^{00}$
must be considered the energy density of the system, then $\int T^{00}dV$
is the total energy o the system. We can multiply our integral by
a constant and the integral is still conserved, we set this constant
to $1/c$, so $P^{0}$ is equal to the energy of the system multiplied
by $1/c$, therefore we get four momentum of the system expression
\[
P^{\alpha}=\frac{1}{c}\int T^{\alpha\beta}dS_{\beta},
\]
which agrees with the expression obtained solving the Killing equation,
see \cite{DEWITT}. 

From the above section we have that we can define the angular momentum
as
\[
J^{\alpha\beta}=\frac{1}{c}\int\left(x^{\alpha}T^{\beta\sigma}-x^{\alpha}T^{\beta\sigma}\right)dS_{\sigma},
\]
lets see that the law of conservation of angular momentum implies
symmetric index in $T^{\alpha\beta}$, this law of conservation is
given by
\[
\partial_{\sigma}\left(x^{\alpha}T^{\beta\sigma}-x^{\alpha}T^{\beta\sigma}\right)=0,
\]
then
\begin{align*}
\delta_{\sigma}^{\alpha}T^{\beta\sigma}+x^{\alpha}\partial_{\sigma}T^{\beta\sigma}-\delta_{\sigma}^{\beta}T^{\alpha\sigma}-x^{\beta}\partial_{\sigma}T^{\alpha\sigma}= & \ \delta_{\sigma}^{\alpha}T^{\beta\sigma}-\delta_{\sigma}^{\beta}T^{\alpha\sigma}\\
= & \ T^{\alpha\beta}-T^{\beta\alpha}\\
= & 0,
\end{align*}
therefore $T^{\alpha\beta}=T^{\beta\alpha}$.

%\chapter{Anexo: Nombrar el anexo C de acuerdo con su contenido}

\end{appendix}