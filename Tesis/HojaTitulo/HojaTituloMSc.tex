%\newpage
%\setcounter{page}{1}
\begin{center}
\begin{figure}
\centering%
\epsfig{file=HojaTitulo/EscudoUN.eps,scale=1}%
\end{figure}
\thispagestyle{empty} \vspace*{2.0cm} \textbf{\huge
Retardo Cosmológico Temporal en Teorías de Gravedad Modificada $f(R)$}\\[6.0cm]
\Large\textbf{Juli\'an Orlando Jim\'enez C\'ardenas}\\[5.5cm]
\small Universidad Nacional de Colombia\\
Facultad de Ciencias, Departamento de F\'isica\\
Bogot\'a D. C. , Colombia\\
%A\~{n}o\\
2019\\
\end{center}

\newpage{\pagestyle{empty}\cleardoublepage}
%%%%%%% titulo
\newpage
\begin{center}
\thispagestyle{empty} \vspace*{0cm} \textbf{\huge
Retardo Cosmológico Temporal en Teorías de Gravedad Modificada $f(R)$}\\[3.0cm]
\Large\textbf{Juli\'an Orlando Jim\'enez C\'ardenas}\\[3.0cm]
\small Tesis o trabajo de grado presentada(o) como requisito parcial para optar al
t\'{\i}tulo de:\\
\textbf{F\'isico}\\[2.5cm]
Director(a):\\
Ph.D. Leonardo Casta\~{n}eda Colorado\\[2.0cm]
L\'{\i}nea de Investigaci\'{o}n: \\
Astrof\'isica, Gravitaci\'on y Cosmologia\\
Grupo de Investigaci\'{o}n:\\
Grupo de Galaxias, Gravitaci\'on y Cosmologia\\[2.5cm]
Universidad Nacional de Colombia\\
Facultad de Ciencias, Departamento de F\'isica\\
Bogot\'a D. C. ,Colombia\\
2019\\
\end{center}

\newpage{\pagestyle{empty}\cleardoublepage}

\newpage
\thispagestyle{empty} \textbf{}\normalsize
\\\\\\%
%%%%%%%%%%%%%%%%%%%%%%%%%%%%%%%%%%%%%%%%%%%%%
%%%%%%%%%%%%%%%%DEDICATORIA%%%%%%%%%%%%%%%%%%%%
%%%%%%%%%%%%%%%%%%%%%%%%%%%%%%%%%%%%%%%%%%%%%
%\textbf{(Dedicatoria o un lema)}\\[4.0cm]

%\begin{flushright}
%\begin{minipage}{8cm}
   % \noindent
     %   \small
       % Su uso es opcional y cada autor podr\'{a} determinar la distribuci\'{o}n del texto en la p\'{a}gina, se sugiere esta presentaci\'{o}n. En ella el autor dedica su trabajo en forma especial a personas y/o entidades.\\[1.0cm]\\
        %Por ejemplo:\\[1.0cm]
        %A mis padres\\[1.0cm]\\
        %o\\[1.0cm]
        %La preocupaci\'{o}n por el hombre y su destino siempre debe ser el
        i%nter\'{e}s primordial de todo esfuerzo t\'{e}cnico. Nunca olvides esto
        %entre tus diagramas y ecuaciones.\\\\
        %Albert Einstein\\
%\end{minipage}
%\end{flushright}

%\newpage{\pagestyle{empty}\cleardoublepage}

%\newpage
%\thispagestyle{empty} \textbf{}\normalsize
%\\\\\\%
%%%%%%%%%%%%%%%%%%%%%%%%%%%%%%%%%%%%%%%%%%%%%
%%%%%%%%%%%%%%AGRADECIMIENTOS%%%%%%%%%%%%%%%%%%%
%%%%%%%%%%%%%%%%%%%%%%%%%%%%%%%%%%%%%%%%%%%%%
%\textbf{\LARGE Agradecimientos}
%\addcontentsline{toc}{chapter}{\numberline{}Agradecimientos}\\\\
%Esta secci\'{o}n es opcional, en ella el autor agradece a las personas o instituciones que colaboraron en la realizaci\'{o}n de la tesis  o trabajo de investigaci\'{o}n. Si se incluye esta secci\'{o}n, deben aparecer los nombres completos, los cargos y su aporte al documento.\\

%\newpage{\pagestyle{empty}\cleardoublepage}

%\newpage
%%%%%%%%%%%%%%%%%%%%%%%%%%%%%%%%%%%%%%%%%%%%%
%%%%%%%%%%%%%%%%%%RESUMEN%%%%%%%%%%%%%%%%%%%%%
%%%%%%%%%%%%%%%%%%%%%%%%%%%%%%%%%%%%%%%%%%%%%
\textbf{\LARGE Resumen}
\addcontentsline{toc}{chapter}{\numberline{}Resumen}\\\\
En la primera parte de este texto se presenta una introducci\'on a la geometr\'ia diferencial como herramienta matem\'atica para la Relatividad General. Se estudia la gravedad linealizada y el papel que esta desempe\~na en la radiaci\'on gravitacional, profundizando as\'i en los conceptos de gauge, energía y contribución cuadrupolar. Seguidamente se presentan las ecuaciones de Einstein relajadas como una generalización para el estudio de la radiación gravitacional y se obtienen expresiones generales para la energía, el momentum lineal y angular. Posteriormente se muestra la relación entre las expresiones de flujo de energía, momentum lineal y angular con el tensor de Weyl.\\[2.0cm]
\textbf{\small Palabras clave: Radiación Gravitacional, Gravedad Linealizada, Ecuaciones de Einstein relajadas, Tensor de Weyl}\\[1.0cm]
%%%%%%%%%%%%%%%%%%%%%%%%%%%%%%%%%%%%%%%%%%%%%
%%%%%%%%%%%%%%%%%%ABSTRACT%%%%%%%%%%%%%%%%%%%%
%%%%%%%%%%%%%%%%%%%%%%%%%%%%%%%%%%%%%%%%%%%%%
\textbf{\LARGE Abstract}\\\\
In the first part of this text an introduction to differential geometry is presented as a tool for General Relativity. The linearized gravity is studied and the role that this one plays in the gravitational radiation, deepening in the gauge, energy and quadrupolar contribution concepts. After this the relaxed Einstein equations are presented as a generalization for the study of gravitational radiation and general expression of energy, linear and angular momentum are obtained. Later it is shown the relation between the flux of energy, lineal and angular momentum with the Weyl tensor.\\[2.0cm]
\textbf{\small Keywords: Gravitational Radiation, Linearized Gravity, Relaxed Einstein field equations, Weyl tensor}\\[1.0cm]