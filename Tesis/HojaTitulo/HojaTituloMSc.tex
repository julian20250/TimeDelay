%\newpage
%\setcounter{page}{1}
\begin{center}
\begin{figure}
\centering%
\epsfig{file=HojaTitulo/EscudoUN.eps,scale=1}%
\end{figure}
\thispagestyle{empty} \vspace*{2.0cm} \textbf{\huge
Retardo Cosmológico Temporal en Modelos de Energía Oscura}\\[6.0cm]
\Large\textbf{Juli\'an Orlando Jim\'enez C\'ardenas}\\[5.5cm]
\small Universidad Nacional de Colombia\\
Facultad de Ciencias, Departamento de F\'isica\\
Bogot\'a D. C. , Colombia\\
%A\~{n}o\\
2019\\
\end{center}

\newpage{\pagestyle{empty}\cleardoublepage}
%%%%%%% titulo
\newpage
\begin{center}
\thispagestyle{empty} \vspace*{0cm} \textbf{\huge
Retardo Cosmológico Temporal en Modelos de Energía Oscura}\\[3.0cm]
\Large\textbf{Juli\'an Orlando Jim\'enez C\'ardenas}\\[3.0cm]
\small Tesis o trabajo de grado presentada(o) como requisito parcial para optar al
t\'{\i}tulo de:\\
\textbf{F\'isico}\\[2.5cm]
Director(a):\\
Ph.D. Leonardo Casta\~{n}eda Colorado\\[2.0cm]
L\'{\i}nea de Investigaci\'{o}n: \\
Astrof\'isica, Gravitaci\'on y Cosmologia\\
Grupo de Investigaci\'{o}n:\\
Grupo de Galaxias, Gravitaci\'on y Cosmologia\\[2.5cm]
Universidad Nacional de Colombia\\
Facultad de Ciencias, Departamento de F\'isica\\
Bogot\'a D. C. ,Colombia\\
2019\\
\end{center}

\newpage{\pagestyle{empty}\cleardoublepage}

\newpage
\thispagestyle{empty} \textbf{}\normalsize
\\\\\\%
%%%%%%%%%%%%%%%%%%%%%%%%%%%%%%%%%%%%%%%%%%%%%
%%%%%%%%%%%%%%%%DEDICATORIA%%%%%%%%%%%%%%%%%%%%
%%%%%%%%%%%%%%%%%%%%%%%%%%%%%%%%%%%%%%%%%%%%%
%\textbf{(Dedicatoria o un lema)}\\[4.0cm]

%\begin{flushright}
%\begin{minipage}{8cm}
   % \noindent
     %   \small
       % Su uso es opcional y cada autor podr\'{a} determinar la distribuci\'{o}n del texto en la p\'{a}gina, se sugiere esta presentaci\'{o}n. En ella el autor dedica su trabajo en forma especial a personas y/o entidades.\\[1.0cm]\\
        %Por ejemplo:\\[1.0cm]
        %A mis padres\\[1.0cm]\\
        %o\\[1.0cm]
        %La preocupaci\'{o}n por el hombre y su destino siempre debe ser el
        i%nter\'{e}s primordial de todo esfuerzo t\'{e}cnico. Nunca olvides esto
        %entre tus diagramas y ecuaciones.\\\\
        %Albert Einstein\\
%\end{minipage}
%\end{flushright}

%\newpage{\pagestyle{empty}\cleardoublepage}

%\newpage
%\thispagestyle{empty} \textbf{}\normalsize
%\\\\\\%
%%%%%%%%%%%%%%%%%%%%%%%%%%%%%%%%%%%%%%%%%%%%%
%%%%%%%%%%%%%%AGRADECIMIENTOS%%%%%%%%%%%%%%%%%%%
%%%%%%%%%%%%%%%%%%%%%%%%%%%%%%%%%%%%%%%%%%%%%
%\textbf{\LARGE Agradecimientos}
%\addcontentsline{toc}{chapter}{\numberline{}Agradecimientos}\\\\
%Esta secci\'{o}n es opcional, en ella el autor agradece a las personas o instituciones que colaboraron en la realizaci\'{o}n de la tesis  o trabajo de investigaci\'{o}n. Si se incluye esta secci\'{o}n, deben aparecer los nombres completos, los cargos y su aporte al documento.\\

%\newpage{\pagestyle{empty}\cleardoublepage}

%\newpage
%%%%%%%%%%%%%%%%%%%%%%%%%%%%%%%%%%%%%%%%%%%%%
%%%%%%%%%%%%%%%%%%RESUMEN%%%%%%%%%%%%%%%%%%%%%
%%%%%%%%%%%%%%%%%%%%%%%%%%%%%%%%%%%%%%%%%%%%%
\textbf{\LARGE Resumen}
\addcontentsline{toc}{chapter}{\numberline{}Resumen}\\\\
En la primera parte de este texto se presenta una introducción a la teoría de la relatividad general, hasta las ecuaciones de campo de Einstein. Posteriormente se discuten algunas consecuencias de la relatividad general en la teoría de lentes gravitacionales hasta llegar a la deducción del tiempo de retraso entre dos imágenes. La penúltima parte de este trabajo refiere a las herramientas estadísticas necesarias para ajustar los parámetros cosmológicos a partir de un conjunto de datos experimentales de tiempos de retraso, y la última parte expone los resultados obtenidos. \\[2.0cm]
\textbf{\small Palabras clave: Tiempo de Retraso, Cadenas de Markov, Modelo Isotermal, Distancias Cosmológicas}\\[1.0cm]
%%%%%%%%%%%%%%%%%%%%%%%%%%%%%%%%%%%%%%%%%%%%%
%%%%%%%%%%%%%%%%%%ABSTRACT%%%%%%%%%%%%%%%%%%%%
%%%%%%%%%%%%%%%%%%%%%%%%%%%%%%%%%%%%%%%%%%%%%
\textbf{\LARGE Abstract}\\\\
In the first part of this text an introduction to the General Relativity theory is presented, until the Einstein's Field Equations. Later on, some consequences of the General Relativity related to the lens theory are discused, until the deduction of the time delay between two images. The penultimate part refers to the statistical tools used to adjust the cosmological parameters via an experimental set of time delays, and the last part presents the results obtained.\\[2.0cm]
\textbf{\small Keywords: Time Delay, Markov Chains, Isothermal Model, Cosmological Distances}\\[1.0cm]