\chapter{Relatividad General}
\section{Introducción}
Este capítulo es una breve introducción a la teoría de la relatividad general, partiendo desde el concepto de variedad, el espacio tangente y cotangente, el concepto de curvatura y el papel que juega la gravedad en todas estas ideas matemáticas. Las referencias clave de este capítulo son \cite{Carroll},\cite{Munkres}.
\section{Variedades}
\subsection{Mapas}
\begin{defi}[Mapa]
	Dados dos conjuntos $A$ y $B$, un mapa $\phi:M\rightarrow N$ es una relación que asigna cada elemento $x\in M$ a un único elemento $y\in N$. En este caso, se denota como $\phi(x)=y$.
\end{defi}
\begin{defi}[Composición de Mapas]
	Con dos mapas $\phi:A\rightarrow B$ y $\Psi:B\rightarrow C$, se define la composición de ambos mapas, $\Psi\circ\phi:A\rightarrow C$, por su acción sobre los elementos de $A$:
	$$(\Psi\circ\phi)(a)=\Psi(\phi(a)).$$
\end{defi}
\begin{center}
	\begin{tikzcd}
		A \arrow[rr, "\Psi\circ\phi"] \arrow[rd, "\phi"] &                      & C \\
		& B \arrow[ru, "\Psi"] &  
	\end{tikzcd}
\end{center}

Un mapa $\phi:A\rightarrow B$ se dice inyectivo (uno a uno) si cada elemento de $B$ tiene a lo sumo un elemento de $A$ que es mapeado a él. Este mapa se dice sobreyectivo si cada elemento de $B$ tiene al menos un elemento de $A$ mapeado a él. $A$ se conoce como el dominio del mapa $\phi$, y su imagen es
$$\mathop{Im}\phi:=\{y\in B: \exists x\in A \text{ tal que }\phi(x)=y\}.$$
La preimagen de un conjunto $U\subseteq B$ bajo la función $\phi$ se define como
$$\phi^{-1}(U):=\{ x\in A: \exists y \in U \text{ tal que }\phi(x)=y\}.$$
Un mapa $\phi:A\rightarrow B$ que es inyectivo y sobreyectivo a la vez se conoce como invertible (biyectivo). En este caso, se define el mapa inverso $\phi^{-1}:B\rightarrow A$ de modo que se satisfaga que, para todo $y\in B$ $(\phi\circ\phi^{-1})(y)=y.$
\begin{center}
	\begin{tikzcd}
		A \arrow[r, "\phi", bend right] & B \arrow[l, "\phi^{-1}"', bend right]
	\end{tikzcd}
\end{center}

Un mapa $f$ de $\mathbb{R}^m$ a $\mathbb{R}^n$ toma una $m-$tupla $(x^1,x^2,\dots,x^m)$ y la envía a una $n-$tupla $(y^1,y^2,\dots,y^n)$, de modo que se puede pensar como una colección de $n$ funciones $\phi^i$ de $m$ variables:
$$y^i=\phi^i(x^1,\dots,x^m)\text{ con }i=1,\dots,n,$$
de modo que
$$f(x^1,\dots,x^m)=(\phi^1(x^1,\dots,x^m),\dots,\phi^n(x^1,\dots,x^m)).$$
Se referirá a cada una de las funciones $\phi^i$ como $C^p$ si son continuas y $p-$veces diferenciables, y al mapa entero $f:\mathbb{R}^m\rightarrow \mathbb{R}^n$ como $C^p$ si cada uno de los campos escalares $\phi^i, i=1,\dots,n$ es al menos $C^p$.

Un mapa $C^0$ es continuo pero no necesariamente diferenciable, mientras que un mapa $C^\infty$ es continuo y puede ser diferenciado cuantas veces se desee. Los mapas $C^\infty$ se llaman suaves.
\begin{defi}[Difeomorfismo]
	El mapa $\phi:A\rightarrow B$ se conoce como difeomorfismo si es biyectivo, y tanto él como su inversa son $C^\infty$. Se dice entonces que los conjuntos $A$ y $B$ son difeomorfos.
\end{defi}

\subsection{Regla de la cadena}
Si tiene dos mapas $f:\mathbb{R}^m\rightarrow \mathbb{R}^n$ y $g:\mathbb{R}^n\rightarrow\mathbb{R}^l,$ que se componen en $(g\circ f):\mathbb{R}^m\rightarrow\mathbb{R}^l$, represente cada espacio en términos de coordenadas: $x^a$ en $\mathbb{R}^m$, $y^b$ en $\mathbb{R}^n$ y $z^c$ en $\mathbb{R}^l$, donde los índices $a,b,c$ varían sobre los valores apropiados.
\begin{center}
	\begin{tikzcd}
		\mathbb{R}^m \arrow[rd, "f"] \arrow[rr, "g\circ f"] &                              & \mathbb{R}^l \\
		& \mathbb{R}^n \arrow[ru, "g"] &             
	\end{tikzcd}
\end{center}

La regla de la cadena relaciona las derivadas parciales de la composición $(g\circ f)$ con las derivadas parciales de los mapas $f$ y $g$ de la siguiente manera
$$\frac{\partial}{\partial x^a}(g\circ f)^c=\sum_{b=1}^n\frac{\partial f^b}{\partial x^a}\frac{\partial g^c}{\partial y^b}.$$

\subsection{Variedades}
En el capítulo de estabilidad se trató el concepto de variedad como un espacio métrico homeomorfo localmente a la bola abierta. En este capítulo se tomará una variedad más general: la variedad topológica, para lo cual se introducirá la idea de topología, y demás conceptos necesarios en términos de la topología de la variedad.
\begin{defi}[Espacio topologico]
	Tome $A$ como un conjunto arbitrario. $\tau$ es una topología para el conjunto $A$ si satisface las siguientes condiciones:
	\begin{enumerate}
		\item $\emptyset,A\in\tau$,
		\item Si $\{U_\alpha\}_{\alpha\in I}\subset\tau$ es una familia arbitraria de elementos de $\tau$, entonces la unión de toda esta familia pertenece a $\tau$, es decir, $\bigcup_{\alpha\in I} U_\alpha\in\tau$, y
		\item Si $\{U_n\}_{n=1}^m\subset\tau$ es una familia finita de elementos de $\tau$, entonces la intersección de todos sus elementos también es un elemento de $\tau,$ es decir, $\bigcap_{n=1}^m U_n\in\tau$.
	\end{enumerate}
	En este caso se dice que la pareja $(A,\tau)$ es un espacio topológico. Los elementos de $\tau$ se llaman abiertos y sus complementos se llaman cerrados.
\end{defi}
\begin{defi}[Carta o sistema coordenado]
	Considere un espacio topológico $(M,\tau)$. Una carta o sistema coordenado $(U, \phi)$ consiste de un conjunto abierto $U\subset M$, junto con un mapa inyectivo $\phi:U\rightarrow\mathbb{R}^n,$ tal que $\phi(U)$ es abierto en $(\mathbb{R}^n, \tau_u)$\footnote{$\tau_u$ denota la topología usual sobre $\mathbb{R}^n$.}.
\end{defi}
\begin{defi}[Atlas $C^r$]
	Un atlas $C^r$ es una colección indexada de cartas $\{(U_\alpha.\phi_\alpha)\}_{\alpha\in I}$, con $\phi_\alpha$ siendo al menos $C^r$, para todo $\alpha\in I$, que satisface las siguientes condiciones
	\begin{enumerate}
		\item $\bigcup_{\alpha\in I} U_\alpha=M$, es decir, $\{U_\alpha\}_{\alpha\in I}$ es un cubrimiento abierto para $M$ y
		\item si para algunos $\alpha,\beta\in I$ ($\alpha\neq\beta$), $U_\alpha\cap U_\beta\neq\emptyset$, entonces el mapa $(\phi_\alpha\circ\phi_\beta^{-1}):\phi_\beta(U_\alpha\cap U_\beta)\rightarrow\phi_\alpha(U_\alpha\cap U_\beta)$ toma puntos en $\phi_\beta(U_\alpha\cap U_\beta)\subseteq\mathbb{R}^n$ y los envía a puntos en $\phi_\alpha(U_\alpha\cap U_\beta)$, y viceversa. Ambas composiciones deben ser $C^r$. Si se satisface esta condición se dice que los mapas $\phi_\alpha$ y $\phi_\beta$ son compatibles
	\end{enumerate}
	
	Un atlas se dice maximal si contiene todas las posibles cartas compatibles.
\end{defi}
\begin{center}
	\begin{tikzcd}
		U_\alpha\cap U_\beta\subset M \arrow[d, "\phi_\beta"] \arrow[rr, "\phi_\alpha"]                    &  & \phi_\alpha(U_\alpha\cap U_\beta)\subset\mathbb{R}^n \arrow[lld, "\phi_\beta\circ\phi_\alpha^{-1}", bend left] \\
		\phi_\beta(U_\alpha\cap U_\beta)\subset\mathbb{R}^n \arrow[rru, "\phi_\alpha\circ\phi_\beta^{-1}"] &  &                                                                                                               
	\end{tikzcd}
\end{center}
\begin{defi}[$C^r$ Variedad $n-$dimensional]
	Una $C^r$ variedad $n-$dimensional es un espacio topológico $(M,\tau)$ junto con un atlas maximal $C^r$.
\end{defi}
El hecho de que una variedad sea localmente como $\mathbb{R}^n$ (a través de las cartas) introduce la posibilidad de usar herramientas del cálculo real sobre ella. Tome por ejemplo dos $C^\infty$ variedades $(M,\tau_M)$ y $(N,\tau_N)$ de dimensión $m$ y $n$, respectivamente. Por simplicidad, pero sin pérdida de generalidad, tome $\phi:M\rightarrow\mathbb{R}^m$ y $\Psi:N\rightarrow\mathbb{R}^n$ como las cartas coordenadas de $M$ y $N$, respectivamente. Si $f:M\rightarrow N$ es una función entre ambas variedades,

\begin{center}
	\begin{tikzcd}
		M \arrow[rr, "f"]                                                            &  & N \arrow[d, "\Psi"] \\
		\mathbb{R}^m \arrow[rr, "\Psi\circ f \circ\phi^{-1}"] \arrow[u, "\phi^{-1}"] &  & \mathbb{R}^n       
	\end{tikzcd}
\end{center}
se puede introducir el concepto de diferenciación sobre el mapa $f$, construyendo el mapa
$$(\Psi\circ f\circ \phi^{-1}):\mathbb{R}^m\rightarrow\mathbb{R}^n,$$
de modo que el operador $\frac{\partial f}{\partial x^\mu}$ quede definido como
$$\frac{\partial f}{\partial x^\mu}:=\frac{\partial }{\partial x^\mu}(\Psi\circ f\circ \phi^{-1}),$$
donde $\mu=1,\dots,m$.
\subsection{Espacio tangente y cotangente}
Tome $\mathcal{F}$ como el espacio de todas las funciones suaves $f:M\rightarrow\mathbb{R}$ ($\phi^{-1}\circ f$ es de clase $C^\infty$, siendo $\phi$ la carta coordenada de $M$). Cada curva $\gamma:\mathbb{R}\rightarrow M$ que pasa por algún punto $p\in M$ define un operador sobre el espacio, la derivada direccional, que mapea $f$ a $$\frac{\mathrm{d}f}{d\lambda}\Big|_{\lambda: \gamma(\lambda)=p}:=\frac{\mathrm{d}}{d\lambda}(f\circ\gamma)(\lambda)$$ (evaluada en $p$).
\begin{center}
	\begin{tikzcd}
		\mathbb{R} \arrow[r, "\gamma"] \arrow[rr, "f\circ\gamma"', bend right] & M \arrow[r, "f"] & \mathbb{R}
	\end{tikzcd}
\end{center}
\begin{defi}[Espacio tangente]
	El espacio tangente $T_pM$ a un punto $p\in M$ es el espacio de los operadores derivadas direccionales dados por todas las curvas que pasan por el punto $p$. Este espacio resulta ser un espacio vectorial.
\end{defi}
El espacio tangente $T_pM$ posee una base natural, $\{\partial_\mu\}$. Cada uno de estos operadores está definido en términos de la curva generada por la carta coordenada del punto $p$. Es decir, si $(U,\phi)$ es una carta coordenada tal que $p\in U$, se toma $(\phi^{-1})^\mu:\mathbb{R}\rightarrow M$ como la restricción de la función $\phi^{-1}$ a una única variable, $x^\mu$, $\mu=1,\dots,m$, con el objetivo de que esta nueva función sea una curva sobre $M$ que pase por $p$, para que defina la derivada direccional $\partial_\mu.$

Para ver que efectivamente es una base del espacio tangente $T_pM$, considere una variedad $m$-dimensional suave $M$, una carta coordenada $(U,\phi)$, una curva $\gamma:\mathbb{R}\rightarrow M$ y una función $f: M\rightarrow \mathbb{R}$.

\begin{center}
	\begin{tikzcd}
		\mathbb{R} \arrow[rrdd, "\phi\circ\gamma"'] \arrow[rr, "\gamma"] \arrow[rrrr, "f\circ\gamma", bend left] &  & U\subseteq M \arrow[dd, "\phi"', bend right] \arrow[rr, "f"]                       &  & \mathbb{R} \\
		&  &                                                                                    &  &            \\
		&  & \mathbb{R}^m \arrow[uu, "\phi^{-1}"', bend right] \arrow[rruu, "f\circ\phi^{-1}"'] &  &           
	\end{tikzcd}
\end{center}
Si $\lambda$ es el parámetro de la curva $\gamma$, se expande el operador $\frac{\mathrm{d}}{\mathrm{d}\lambda}$ en términos de los operadores $\partial_\mu$ aplicando la regla de la cadena:
$$\frac{\mathrm{d}f}{\mathrm{d}\lambda}=\frac{\mathrm{d}}{\mathrm{d}\lambda}(f\circ\gamma)=\frac{\mathrm{d}}{\mathrm{d}\lambda}((f\circ\phi^{-1})\circ(\phi\circ\gamma))=\frac{\mathrm{d}(\phi\circ\gamma)^\mu}{\mathrm{d}\lambda}\frac{\partial(f\circ\phi^{-1})}{\partial x^\mu}=\frac{\mathrm{d}x^\mu}{\mathrm{d}\lambda}\partial_\mu f.$$
Como la función $f$ es arbitraria,
$$\frac{\mathrm{d}}{\mathrm{d}\lambda}=\frac{\mathrm{d}x^\mu}{\mathrm{d}\lambda}\partial_\mu,$$
con lo que los operadores derivada direccional $\{\partial_\mu\}$ son una base para $T_pM$, conocida como base coordenada. Además, esto implica que el espacio tangente $T_pM$ tiene la misma dimensión de la variedad.

Una de las ventajas de este punto de vista de los vectores como operadores diferenciales es que la ley de transformación es inmediata. Como los vectores de la base son $\hat{e}_{(\mu)}=\partial_{\mu}$, los vectores de la base en un nuevo sistema coordenado $x^{\mu'}$ están dadas por la regla de la cadena \cite{Carroll}

$$\partial_{\mu^{\prime}}=\frac{\partial x^{\mu}}{\partial x^{\mu^{\prime}}} \partial_{\mu}$$

La ley de transformación de vectores se introduce de tal forma que un vector del espacio tangente $V=V^\mu \partial_\mu$ permanezca invariante bajo un cambio de base, es decir,
$$V^\mu\partial_\mu=V^{\mu'}\partial_{\mu'}=V^{\mu'}\frac{\partial x^\mu}{\partial x^{\mu'}}\partial_\mu,$$
y como la matriz $\frac{\partial x^{\mu'}}{\partial x^{\mu}}$ es la inversa de $\frac{\partial x^\mu}{\partial x^{\mu'}}$, la ley de transformación es

\begin{equation}
	V^{\mu'}=\frac{\partial x^{\mu'}}{\partial x^{\mu}} V^\mu.
\end{equation}
\begin{defi}[Espacio cotangente]
	El espacio cotangente $T^*_pM$ de una variedad $M$ en un punto $p\in M$ es el conjunto de los mapas lineales $\omega:T_pM\rightarrow \mathbb{R}$. Los elementos de este espacio se conocen como 1-formas.
\end{defi}

El ejemplo canónico de 1-forma es el gradiente de una función $f:M\rightarrow \mathbb{R}$, denotado por $\mathrm{d}f$. Su acción sobre un vector $\frac{\mathrm{d}}{\mathrm{d}\lambda}$ del espacio tangente es exactamente la derivada direccional sobre la función $f$:
$$\mathrm{d}f\left( \frac{\mathrm{d}}{\mathrm{d}\lambda} \right)=\frac{\mathrm{d}f}{\mathrm{d}\lambda}\Big|_{p}.$$
Justo como las derivadas parciales a lo largo de los ejes coordenados proveen una base natural para el espacio tangente, los gradientes de las funciones coordenadas $x^\mu$ proveen una base natural para el espacio cotangente $\{\mathrm{d}x^\mu\}$, conocida como base dual. Observe que, al aplicar $\mathrm{d}x^\mu$ a $\partial_\eta$ se obtiene que
$$\mathrm{d}x^\mu(\partial_\nu)=\frac{\partial x^\mu}{\partial x^\nu}=\frac{\partial}{\partial x^\nu}(x^\mu\circ (x^\nu)^{-1})=\frac{\partial}{\partial x^\nu}((x^\mu\circ\phi^{-1})\circ(\phi\circ (x^\nu)^{-1})).$$
\begin{center}
\begin{tikzcd}
\mathbb{R} \arrow[rr, "(x^\nu)^{-1}"] \arrow[rrdd, "\phi\circ(x^\nu)^{-1}"'] &  & M \arrow[dd, "\phi"]                                                      &  &                       \\
                                                                             &  &                                                                           &  &                       \\
                                                                             &  & \mathbb{R}^m \arrow[rr, "\phi^{-1}"] \arrow[rrdd, "x^\mu\circ\phi^{-1}"'] &  & M \arrow[dd, "x^\mu"] \\
                                                                             &  &                                                                           &  &                       \\
                                                                             &  &                                                                           &  & \mathbb{R}           
\end{tikzcd}
\end{center}

Aplicando la regla de la cadena,
$$\frac{\partial}{\partial x^\nu}((x^\mu\circ\phi^{-1})\circ(\phi\circ (x^\nu)^{-1}))=\frac{\partial(\phi\circ(x^\nu)^{-1})^\eta}{\partial x^\nu}\frac{\partial(x^\mu\circ\phi^{-1})}{\partial x^\eta}.$$
Intuitivamente, $x^\mu\circ\phi^{-1}=x^\mu,$ donde $x^\mu$ del lado izquierdo de la igualdad es la $\mu$-ésima coordenada de la carta, y al lado derecho es la $\mu$-ésima componente de $\mathbb{R}^m$. Por otro lado, $(\phi\circ(x^\nu)^{-1})^\eta=x^\eta,$ con lo que
$$\frac{\partial(\phi\circ(x^\nu)^{-1})^\eta}{\partial x^\nu}\frac{\partial(x^\mu\circ\phi^{-1})}{\partial x^\eta}=\frac{\partial x^\eta}{\partial x^\nu}\frac{\partial x^\mu}{\partial x^\eta}=\delta_\eta^\mu \frac{\partial x^\eta}{\partial x^\nu}=\frac{\partial x^\mu}{\partial x^\nu}=\delta_\nu^\mu.$$
En resumen,
\begin{equation}
	\boxed{\mathrm{d}x^\mu(\partial_\nu)=\delta_\nu^\mu.}
\end{equation}
Esta condición determina que $\{\mathrm{d}x^\mu\}$ es una base para el espacio cotangente $T^*_p M$ \cite{Carroll}. De este modo, cualquier 1-forma $\omega$ se puede expandir en sus componentes: $\omega=\omega_\mu \mathrm{d}x^\mu$. Las propiedades de transformación de los vectores de la base dual y las componentes de una 1-forma se siguen de la misma forma que en el caso del espacio tangente:
$$\mathrm{d}x^{\mu'}=\frac{\partial x^{\mu'}}{\partial x^\mu} \mathrm{d}x^\mu;\ \omega_{\mu'}=\frac{\partial x^\mu}{\partial x^{\mu'}}\omega_\mu.$$

\begin{defi}[Espacio producto cartesiano]
	Se define el espacio producto cartesiano $\Pi_l^k$ respecto a un punto $p\in M$ de la variedad como:
	$$\Pi_l^k:=\underbrace{T^*_pM\times\cdots\times T^*_pM}_{l-\text{veces}}\times \underbrace{T_pM\times\cdots\times T_pM}_{k-\text{veces}}, \text{ es decir,}$$
	$$\Pi_l^k=\{(\omega^1,\omega^2,\dots,\omega^l,\mathds{X}_1,\mathds{X}_2,\dots,\mathds{X}_k): \omega^i\in T^*_pM; \ \mathds{X}_i\in T_pM\}.$$
	Este espacio es un espacio vectorial con la suma y el producto usuales.
\end{defi}
\begin{defi}[Tensores]
	Un tensor $(k,l)$ $\mathds{T}:\Pi_l^k\rightarrow \mathbb{R}$ es un mapa multilineal (lineal en cada uno de sus argumentos). Este tensor se puede expandir en términos de las bases del espacio tangente y cotangente de la siguiente forma:
	
	$$\mathds{T}=?{T}^{\mu_1\cdots\mu_k}_{\nu_1\cdots{\nu_l}}? \partial_{\mu_1}\otimes \cdots\otimes \partial_{\mu_k}\otimes \mathrm{d}x^{\nu_1}\otimes\cdots\otimes \mathrm{d}x^{\nu_l},$$
	donde $?{T}^{\mu_1\cdots\mu_k}_{\nu_1\cdots{\nu_l}}?$ son los coeficientes del tensor.
\end{defi}
De modo similar al caso de los vectores, los tensores transforman coordenadas en cada uno de sus índices de la siguiente forma:
$$?T^{\mu_1'\cdots\mu_k'}_{\nu_1'\cdots\nu_l'}?=\frac{\partial x^{\mu_1'}}{\partial x^{\mu_1}}\cdots\frac{\partial x^{\mu_k'}}{\partial x^{\mu_k}}\frac{\partial x^{\nu_1}}{\partial x^{\nu_1'}}\cdots \frac{\partial x^{\nu_l}}{\partial x^{\nu_l'}}?{T}^{\mu_1\cdots\mu_k}_{\nu_1\cdots{\nu_l}}?.$$
Infortunadamente, la derivada parcial de un tensor no es, en general, un tensor (no cumple esta regla de transformación de coordenadas). Esto motivará posteriormente la derivada covariante, que preservará el carácter tensorial tras aplicarse sobre un tensor.

\begin{defi}[Tensor métrico]
	El tensor métrico $?g_{\mu\nu}?$ es un tensor simétrico $(0,2)$, cuya representación matricial tiene determinante no nulo ($g=|g_{\mu\nu}|\neq0$), y satisface la relación
	\begin{equation}\label{metricTensor}
		?g^{\mu\nu}? ?g_{\nu\sigma}?= ?\delta^\mu_\sigma?.
	\end{equation}
	La simetría de $?g_{\mu\nu}?$ implica la simetría de $?g^{\mu\nu}?$, y la relación \eqref{metricTensor} permite que el tensor métrico se pueda usar para subir o bajar índices.
\end{defi}
\begin{defi}[Elemento de línea]
	El elemento de línea se define de la siguiente forma
	$$\mathrm{d}s^2=?g_{\mu\nu}? \mathrm{d}x^\mu \mathrm{d}x^\nu.$$
\end{defi}
