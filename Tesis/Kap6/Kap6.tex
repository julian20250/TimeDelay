\chapter{Conclusions and recommendations}
\section{Conclusions}
It was understood the radiation conditions for the linearized gravity, this includes the comparison of the radiation zone in electrodynamics and in linearized gravity, the role of the background and physical manifold, how gauge transformations works and the energy contribution of gravitational radiation and the study of the characteristic length of the background and the radiation. The function of manifolds in gravitational radiation is usually omitted, just like why the energy has to be average and how it can be average. Gauge transformations are not usually study from the point of view of the diffeomorphism between manifolds and the characteristic length is usually just mention. The general equation for the study of gravitational radiation was deduced and the expressions of energy, lineal and angular momentum where obtained. Finally the contribution of the Weyl tensor in gravitational radiation was highlighted, this role is really important for the generation of gravitational radiation for astrophysical systems.
\section{Recommendations}
The idea is to apply this theory into an astrophysical system, be able to calculate the energy contribution of this system, and even try to simulate it. Make a comparison between a simulation made from linearized gravity point of view and the relaxed Einstein field equations, so it can be check at which point linearized gravity is a good approximation and study the Weyl tensor contribution for this system. There exists codes which helps in the analysis of this systems, a particular one is \textit{Einstein Toolkit}\footnote{Einstein Toolkit website: https//einsteintoolkit.org}, which use numerical relativity for its simulations. The study of this kind of codes could improve the knowledge in gravitational radiation for astrophysical systems.\\