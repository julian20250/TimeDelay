\chapter*{Notación}
\addcontentsline{toc}{chapter}{\numberline{}Notación}
En esta sección se expone la notación y convención utilizadas en los diferentes capítulos de este trabajo.
\section*{Relatividad General}
	Se identificarán los índices latinos con las dimensiones espaciales. Por ejemplo,
	$$x^i(\tau)$$
	representa las coordenadas $i=1,2,3$ de la 4-posición. Los índices griegos harán referencia a las dimensiones espacio-temporal. Por ejemplo, cuando se denota
	$$x^\mu(\tau),$$
	se hace referencia a las cuatro componentes de la 4-posición, $\mu=0,1,2,3$. La dimensión temporal será la componente $\mu=0$, y las espaciales serán $\mu=1,2,3$. Se usará la notación de suma de Einstein sobre las componentes espaciales o espacio-temporales; por ejemplo,
	$$g_{\mu\nu}\mathrm{d}x^\mu \mathrm{d}x^\nu\equiv \sum_{\mu=0}^3 \sum_{\nu=0}^3 g_{\mu\nu}\mathrm{d}x^\mu \mathrm{d}x^\nu \text{ y}$$
	$$x^iy^i=\sum_{i=1}^3 x^iy^i.$$
	A menos que se comente lo contrario, se tomará la velocidad de la luz $c=1$.
	
	La signatura que usará para todos los tensores métricos será $(-,+,+,+)$. Por ejemplo, la métrica de Minkowski será, en representación matricial: $\eta=\mathop{diag}(-1,1,1,1)$.
	
	Se denotará como $\epsilon_{ijk}$ al símbolo de Levi-Civita, equivalente a $1$ cuando $abc$ es una permutación par de $123$, $-1$ cuando $abc$ es una permutación impar de $abc$, y $0$ cuando hay al menos dos índices repetidos.