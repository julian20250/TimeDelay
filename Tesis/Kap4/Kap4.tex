\chapter{Chapter 3: Relaxed Einsteins equations \label{chp3}}

In chapter 2 we discuss the nature of a background and a physical
manifold, now we are going to write the Einsteins equations in such
a way that are going to be helpful for the analysis of the measure
of energy, linear and angular momentum carried by gravitational waves. The expressions obtained in this chapter are general expressions, this means that there is no assumptions over the gravitational field. Then a wave equation is obtained for the general case, which can be reduced, in the case of the linearized gravitational field, to the wave equation obtained in Chapter \ref{chp2}. Main references of this chapter \cite{LANDAU,POISSON,DEWITT,THORNE-MP}

\section{Landau-Lifshitz form of Einstein field equations}

The rewriting of the Einsteins field equations arises from the fact
that the equations
\begin{equation}
\nabla_{\beta}T_{\alpha}^{\beta}=\frac{1}{\sqrt{-g}}\partial_{\beta}\left(T_{\alpha}^{\beta}\sqrt{-g}\right)-\frac{1}{2}\left(\partial_{\alpha}g_{\beta\gamma}\right)T^{\beta\gamma}=0\label{eq:tg}
\end{equation}
do not generally express any conservation law whatever, we must include
the conservation of the gravitational field. We choose a coordinate
system such that at some particular point in spacetime all the first
derivatives of $g_{\alpha\beta}$ vanish, so the equation (\ref{eq:tg})
is written as
\[
\partial_{\beta}T^{\alpha\beta}=0.
\]

We are going to show that we ca write $T^{\alpha\beta}$ in the following
form
\begin{equation}
T^{\alpha\beta}=\partial_{\gamma}h^{\alpha\beta\gamma},\label{eq:th}
\end{equation}
where $h^{\alpha\beta\gamma}$ is such that
\[
h^{\alpha\beta\gamma}=-h^{\alpha\gamma\beta}.
\]
In order to do this we start from
\begin{equation}
T^{\alpha\beta}=\frac{c^{4}}{8\pi G}\left(R^{\alpha\beta}-\frac{1}{2}g^{\alpha\beta}R\right),\label{eq:tt}
\end{equation}
because we are in a particular coordinate system we write the tensor
$R^{\alpha\beta}$ as
\[
R^{\alpha\beta}=\frac{1}{2}g^{\alpha\sigma}g^{\beta\rho}g^{\delta\gamma}\left(\partial_{\sigma}\partial_{\gamma}g_{\delta\rho}+\partial_{\delta}\partial_{\rho}g_{\sigma\gamma}-\partial_{\sigma}\partial_{\rho}g_{\delta\gamma}-\partial_{\delta}\partial_{\delta}g_{\sigma\rho}\right),
\]
therefore the equation (\ref{eq:tt}) takes the form
\[
T^{\alpha\beta}=\partial_{\sigma}\left\{ \frac{c^{4}}{16\pi G\left(-g\right)}\partial_{\mu}\left[\left(-g\right)\left(g^{\alpha\beta}g^{\sigma\mu}-g^{\alpha\sigma}g^{\beta\mu}\right)\right]\right\} .
\]
Let us define
\begin{equation}
H^{\alpha\beta\mu\nu}=\frac{c^{4}}{16\pi G}\left(-g\right)\left(g^{\alpha\beta}g^{\mu\nu}-g^{\alpha\mu}g^{\beta\nu}\right),\label{eq:H}
\end{equation}
therefore according to (\ref{eq:th})
\[
h^{\alpha\beta\mu}=\partial_{\nu}H^{\alpha\beta\mu\nu},
\]
then we write our field equations as
\[
\partial_{\mu}\partial_{\nu}H^{\alpha\beta\mu\nu}=\left(-g\right)T^{\alpha\beta}.
\]

This relations face a problem, is that not allways we are in a coordinate
system such that all the first derivatives of $g_{\alpha\beta}$ vanish,
then in the general case 
\[
\partial_{\mu}\partial_{\nu}H^{\alpha\beta\mu\nu}-\left(-g\right)T^{\alpha\beta}\neq0.
\]
Let us denote this difference as $\left(-g\right)t_{LL}^{\alpha\beta}$,
where $t_{LL}^{\alpha\beta}$ is symmetric in $\alpha$ and $\beta$,
then
\begin{equation}
\partial_{\mu}\partial_{\nu}H^{\alpha\mu\beta\nu}=\left(-g\right)\left(T^{\alpha\beta}+t_{LL}^{\alpha\beta}\right),\label{eq:ll-eq}
\end{equation}
from (\ref{eq:tt}) and (\ref{eq:H}) we obtain an expression of $t^{\alpha\beta}$
in terms of $g^{\alpha\beta}$ and $\mathfrak{g}^{\alpha\beta}$,
where 
\begin{equation}
\mathfrak{g}^{\alpha\beta}=\sqrt{-g}g^{\alpha\beta},\label{eq:goth-g}
\end{equation}
 then
\begin{align*}
\left(-g\right)t_{LL}^{\alpha\beta}= & \ \frac{c^{4}}{16\pi G}\left\{ \partial_{\lambda}\mathfrak{g}^{\alpha\beta}\partial_{\mu}\mathfrak{g}^{\lambda\mu}-\partial_{\lambda}\mathfrak{g}^{\alpha\lambda}\partial_{\mu}\mathfrak{g}^{\beta\mu}+\frac{1}{2}g^{\alpha\beta}g_{\lambda\mu}\partial_{\rho}\mathfrak{g}^{\lambda\nu}\partial_{\nu}\mathfrak{g}^{\mu\rho}\right.\\
\  & \ g^{\alpha\lambda}g_{\mu\nu}\partial_{\rho}\mathfrak{g}^{\beta\nu}\partial_{\lambda}\mathfrak{g}^{\mu\rho}-g^{\beta\lambda}g_{\mu\nu}\partial_{\rho}\mathfrak{g}^{\alpha\nu}\partial_{\lambda}\mathfrak{g}^{\mu\rho}+g_{\lambda\mu}g^{\nu\rho}\partial_{\nu}\mathfrak{g}^{\alpha\lambda}\partial_{\rho}\mathfrak{g}^{\mu\beta}\\
\  & \ \left.\frac{1}{8}\left(2g^{\alpha\lambda}g^{\beta\mu}-g^{\alpha\beta}g^{\lambda\mu}\right)\left(2g_{\nu\rho}g_{\sigma\tau}-g_{\rho\sigma}g_{\nu\tau}\right)\partial_{\lambda}\mathfrak{g}^{\nu\tau}\partial_{\mu}\mathfrak{g}^{\rho\sigma}\right\} .
\end{align*}
This is known as the Landau-Lifshitz pseudotensor, and the expression
(\ref{eq:ll-eq}) is known as the Landau-Lifshitz form of the Einsteins
field equations. From equation (\ref{eq:ll-eq}) we have, finally,
our conservation equation
\begin{equation}
\partial_{\beta}\left[\left(-g\right)\left(T^{\alpha\beta}+t^{\alpha\beta}\right)\right]=0.\label{eq:gen-con-law}
\end{equation}

In the above analysis we supposed that we had a Minkowskian background
metric, thats why appeared partial derivatives. In the case where
we have a background metric $\bar{g}_{\alpha\beta}$ we must change
the derivative respect to the flat background by the covariant derivative,
$\partial_{\mu}\rightarrow\bar{\nabla}_{\mu}$. 

\section{Linear and angular momentum}

\subsection{In Flat spacetime}

Let $\xi^{\alpha}$ be a Killing vector , therefore satisfies the
Killing equation
\[
\nabla_{\alpha}\xi_{\beta}+\nabla_{\beta}\xi_{\alpha}=0,
\]
we can build conserved quantities out of $T^{\alpha\beta}$, then
\begin{align*}
\nabla_{\beta}\left(\xi_{\alpha}T^{\alpha\beta}\right)= & \ T^{\alpha\beta}\nabla_{\beta}\xi_{\alpha}+\xi_{\alpha}\nabla_{\beta}T^{\alpha\beta}\\
= & \ \frac{1}{2}T^{\alpha\beta}\nabla_{\beta}\xi_{\alpha}+\frac{1}{2}T^{\beta\alpha}\nabla_{\alpha}\xi_{\beta}\\
= & \ \frac{1}{2}\left(\nabla_{\alpha}\xi_{\beta}+\nabla_{\beta}\xi_{\alpha}\right)T^{\alpha\beta}=0,
\end{align*}
implying the conservation of
\begin{equation}
\int\xi_{\alpha}T^{\alpha\beta}dS_{\beta}\label{eq:3cons}
\end{equation}
where we are integrating over any hypersurface. If we do this for
the flat spacetime, we have the Killing equation
\[
\partial_{\alpha}\xi_{\beta}+\partial_{\beta}\xi_{\alpha}=0,
\]
which has the general solution
\begin{equation}
\label{eq:killing-general}
\xi_{\alpha}=a_{\alpha}+l_{\alpha\beta}x^{\beta}
\end{equation}
where $a_{\alpha}$ and $l_{\alpha\beta}$ are constants with $l_{\alpha\beta}=-l_{\beta\alpha}$.
From equation (\ref{eq:3cons}) the corresponding conserved quantity
for $\xi_{\alpha}$ is
\[
\int\left(a_{\alpha}+l_{\alpha\beta}x^{\beta}\right)T^{\alpha\sigma}dS_{\sigma}=a_{\alpha}P^{\alpha}-\frac{1}{2}l_{\alpha\beta}J^{\alpha\beta}
\]
given that $a_{\alpha}$ and $l_{\alpha\beta}$ are constants, we
set them in such a way that
\begin{align*}
P^{\alpha}= & \ \frac{1}{c}\int T^{\alpha\beta}dS_{\beta},\\
J^{\alpha\beta}= & \ \frac{1}{c}\int\left(x^{\alpha}T^{\beta\sigma}-x^{\alpha}T^{\beta\sigma}\right)dS_{\sigma},
\end{align*}
where $P^{\alpha}$ the total energy momentum four vector and $J^{\alpha\beta}$
the total angular momentum tensors, we see that these agrees with
the expression obtained in Appendix \ref{appendix-momentum}

\subsection{General expressions}

Equation (\ref{eq:gen-con-law}) shows us a conservation law equation,
from this we are going to obtain expressions for the linear and angular
momentum. Let us define the quantities
\[
P^{\alpha}=\frac{1}{c}\int\left(-g\right)\left(T^{\alpha\beta}+t^{\alpha\beta}\right)dS_{\beta},
\]
where $S_{\beta}$ is an hypersurface, $P^{\alpha}$ this is known
as the total four-momentum of matter plus gravitational field. This
integration can be taken over any infinite hypersurface, including
all of the three-dimensional space. If we choose for this the hypersurface
$x^{0}=\text{const}$, then $P^{\alpha}$ can be written in the form
of a three-dimensional space integral
\begin{equation}
P^{\alpha}=\frac{1}{c}\int\left(-g\right)\left(T^{\alpha0}+t^{\alpha0}\right)dV.\label{eq:p-general}
\end{equation}
The fact that $T^{\alpha\beta}+t^{\alpha\beta}$ has symmetric index
implies that there is a conservation law for the angular momentum \footnote{See Appendix \ref{appendix-momentum}},
which is defined as
\[
J^{\alpha\beta}=\frac{1}{c}\int\left(-g\right)\left[x^{\alpha}\left(T^{\beta\sigma}+t^{\beta\sigma}\right)-x^{\beta}\left(T^{\alpha\sigma}+t^{\alpha\sigma}\right)\right]dS_{\sigma}.
\]

As we saw in Chapter 2, the expression (\ref{eq:p-general}) is used
to obtain the energy flux and the momentum flux for the gravitational
radiation, therefore there must be and expression for the flux of
angular momentum for the gravitational waves in the local wave zone,
this expression was obtained in \cite{THORNE-MP}. In the TT gauge this expression
is
\begin{equation}
\label{eq:an-mom-flux}
\frac{dJ^{i}}{dt}=\frac{r^{2}}{32\pi c^{2}}\int\epsilon^{ijk}\left(x_{j}\partial_{k}h_{ab}+2\delta_{aj}h_{bk}\right)\partial_{r}h^{ab}d\Omega,
\end{equation}
where $\epsilon^{ijk}$ is the three-dimensional Levi-Civita antisymmetric
tensor. This expression is going to be useful in the next chapter.

\section{Relaxed Einstein equations}

\subsection{Harmonic coordinates and a wave equation}

We want to reach a wave equation form for de Einstein field equations,
just like in (\ref{efh}), but without any assumptions over the dimensions
of $g_{\alpha\beta}$. We define the field $\text{h}^{\alpha\beta}$
such that
\begin{equation}
\text{h}^{\alpha\beta}=\eta^{\alpha\beta}-\mathfrak{g}^{\alpha\beta}\label{eq:h-general}
\end{equation}
this is indeed a expression that generalize our linearized gravitational
field\footnote{For details of this discussion see \cite{THORNE-GR}}, if we
assume that $g_{\alpha\beta}=\eta_{\alpha\beta}+h_{\alpha\beta}+\mathcal{O}\left(h^{2}\right)$
then, taking $h_{\alpha\beta}$ only to linear order
\begin{align*}
\text{h}^{\alpha\beta}\approx & \ \eta^{\alpha\beta}-\sqrt{1+h}\left(\eta^{\text{\ensuremath{\alpha\beta}}}-h^{\alpha\beta}\right)\\
= & \ h^{\alpha\beta}-\frac{1}{2}\eta^{\alpha\beta}h,
\end{align*}
then $\text{h}^{\alpha\beta}$ reduce to $\bar{h}^{\alpha\beta}$. 

We will write our wave equation using
the Landau-Lifshitz form of Einstein field equations. We now impose
the DeDonder gauge
\begin{equation}
\partial_{\beta}\text{h}^{\alpha\beta}=0,\label{eq:gauge-h}
\end{equation}
calculating $H^{\alpha\mu\beta\nu}$ using (\ref{eq:H}), (\ref{eq:h-general})
and (\ref{eq:goth-g})
\[
H^{\alpha\mu\beta\nu}=\left(\eta^{\alpha\beta}\eta^{\mu\nu}-\eta^{\alpha\beta}\text{h}^{\mu\nu}-\text{h}^{\alpha\beta}\eta^{\mu\nu}+\text{h}^{\alpha\beta}\text{h}^{\mu\nu}\right)-\left(\eta^{\alpha\nu}\eta^{\beta\mu}-\eta^{\alpha\nu}\text{h}^{\beta\mu}-\text{h}^{\alpha\nu}\eta^{\beta\mu}+\text{h}^{\alpha\nu}\text{h}^{\beta\mu}\right),
\]
derivating with respect to $x^{\nu}$ and $x^{\mu}$ and using (\ref{eq:gauge-h})
\[
\partial_{\mu}\partial_{\nu}H^{\alpha\mu\beta\nu}=-\square\text{h}^{\alpha\beta}+\text{h}^{\mu\nu}\partial_{\mu}\partial_{\nu}\text{h}^{\alpha\beta}-\partial_{\mu}\text{h}^{\alpha\nu}\partial_{\nu}\text{h}^{\beta\mu},
\]
let us define the pseudotensor $t_{H}^{\alpha\beta}$ such that
\[
\left(-g\right)t_{H}^{\alpha\beta}=\frac{c^{4}}{16\pi G}\left(\partial_{\mu}h^{\alpha\nu}\partial_{\nu}h^{\beta\mu}-h^{\mu\nu}\partial_{\mu}\partial_{\nu}h^{\alpha\beta}\right),
\]
and the pseudotensor $\tau^{\alpha\beta}$ in the following way
\[
\tau^{\alpha\beta}=\left(-g\right)\left(T^{\alpha\beta}+t_{LL}^{\alpha\beta}+t_{H}^{\alpha\beta}\right),
\]
this tensor is known as the \textbf{effective energy-momentum pseudotensor}.
With this we have the following wave equation
\begin{equation}
\square\text{h}^{\alpha\beta}=-\frac{16\pi G}{c^{4}}\tau^{\alpha\beta}.\label{eq:einstein-wave}
\end{equation}
The pseudotensor $t_{H}^{\alpha\beta}$ have the property that $\partial_{\beta}\left[\left(-g\right)t_{H}^{\alpha\beta}\right]=0$,
this because, using (\ref{eq:gauge-h})
\begin{align*}
\partial_{\beta}t_{H}^{\alpha\beta} & =\partial_{\beta}\partial_{\mu}\text{h}^{\alpha\nu}\cdot\partial_{\nu}\text{h}^{\beta\mu}-\partial_{\beta}\text{h}^{\mu\nu}\partial_{\mu}\partial_{\nu}\text{h}^{\alpha\beta}\\
\  & =\partial_{\nu}\partial_{\mu}\text{h}^{\alpha\beta}\cdot\partial_{\beta}\text{h}^{\nu\mu}-\partial_{\beta}\text{h}^{\mu\nu}\partial_{\mu}\partial_{\nu}\text{h}^{\alpha\beta}\\
\  & =0,
\end{align*}
therefore
\begin{equation}
\partial_{\beta}\tau^{\alpha\beta}=0.\label{eq:cons-tau}
\end{equation}

The name of relaxed equations is because the equation (\ref{eq:einstein-wave})
can be solved without enforcing the gauge condition (\ref{eq:gauge-h})
or the conservation statement (\ref{eq:cons-tau}).

\subsection{Formal solution to the wave equation}

As the linearized equation, this can be solved by the method of Green's
function: if $G(\boldsymbol{x}-\boldsymbol{x}')$ is a solution of
the equation
\[
\boxempty_{\boldsymbol{x}}G\left(\boldsymbol{x}-\boldsymbol{x}'\right)=\delta^{4}\left(\boldsymbol{x}-\boldsymbol{x}'\right),
\]
where $\boxempty_{\boldsymbol{x}}$ is the d'Alembertian operator
with derivatives taken with respect to the variable $\boldsymbol{x}$,
then the corresponding solution of equation (\ref{eq:einstein-wave})
is
\[
\text{h}^{\alpha\beta}=\frac{4G}{c^{4}}\int G\left(\boldsymbol{x}-\boldsymbol{x}'\right)\tau^{\alpha\beta}\left(\boldsymbol{x}'\right)d^{4}\boldsymbol{x}'.
\]
Just like in the linearized case, the appropriate solution is the
retarded Green's function
\[
G\left(\boldsymbol{x}-\boldsymbol{x}'\right)=-\frac{1}{4\pi\left|\boldsymbol{x}-\boldsymbol{x}'\right|}\delta\left(x_{\text{ret}}^{0}-{x'}^{0}\right).
\]

For details of the calculation of the Green's function see Box 6.5 of \cite{POISSON}.