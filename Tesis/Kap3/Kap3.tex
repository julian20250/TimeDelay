\chapter{Estadística bayesiana}
En este capítulo se hace una breve introducción a la teoría bayesiana, con el fin de obtener las herramientas necesarias para ajustar los parámetros de un modelo dado a través de un conjunto de datos experimentales. Algunas referencias clave de este capítulo son \cite{Arunachalam}, \cite{rossi_2018}.
\section{Preliminares}
Antes de enunciar la regla de Bayes, es necesario definir lo que es un espacio de probabilidad, y la probabilidad condicional. Esta sección busca fundamentar las bases de la teoría de la probabilidad, para comprender a cabalidad temas posteriores.
\subsection{Espacio de probabilidad}
La probabilidad es una teoría matemática que busca medir de alguna forma la posibilidad de que ocurra un evento contenido en un conjunto de posibles eventos, resultados todos de un experimento. Por supuesto, no se conoce de forma determinista cuál será el resultado tras la ejecución del experimento, de modo que sólo se puede hablar de posibilidades de que ocurra algún evento. Este tipo de experimentos se conocerán como experimentos aleatorios.

\begin{defi}[Experimento Aleatorio]
		Un experimento se dice aleatorio si su resultado no se puede determinar de antemano.
	\end{defi}
\begin{defi}[Espacio de Muestra]
		El conjunto $\Omega$ de todos los posibles resultados de un experimento aleatorio se llama espacio de muestra. Un elemento $\omega\in\Omega$ se llama resultado o muestra. $\Omega$ se dice discreto si es finito o contable.
	\end{defi}	
	Ahora se requiere definir lo que se entenderá por evento, para lo cual se definirá una estructura sobre el espacio de muestra, conocida como $\sigma-$álgebra, que dará cuenta de los eventos de interés tras la ejecución de un experimento aleatorio. 
	\begin{defi}[$\sigma$-álgebra]
		Tome $\Omega\neq\emptyset$. Una colección $\Im$ de subconjuntos de $\Omega$ se llama $\sigma-$álgebra sobre $\Omega$ si:
		\begin{enumerate}
			\item $\Omega\in\Im$,
			\item Si $A\in\Im$, entonces $A^c\in\Im$ y,
			\item Si $A_1,A_2,\dots\in\Im$, entonces $\bigcup_{i=1}^\infty A_i\in\Im$.
		\end{enumerate}
		Los elementos de $\Im$ se llaman eventos.
	\end{defi}
	El siguiente teorema será de utilidad para construir una $\sigma$-álgebra a partir de un conjunto finito o contable de $\sigma$-álgebras.
	\begin{theo}
		Si $\Omega\neq\emptyset$ y $\Im_1,\Im_2,\dots$ son $\sigma-$álgebras sobre $\Omega$, entonces $\bigcap_{i=1}^\infty \Im_i$ es una $\sigma$-álgebra sobre $\Omega$.
	\end{theo}		
	
	\begin{proof}
		Como $\Omega\in\Im_j,$ para $j=1,2,\dots$, $\Omega\in\bigcap_{j=1}^\infty \Im_j$. Si $A\in\bigcap_{j=1}^\infty\Im_j$, $A\in\Im_j$, para $j=1,2,\dots$, de modo que $A^c\in\Im_j$, y $A^c\in\bigcap_{j=1}^\infty\Im_j$. Por último, si 
		$$A_1,A_2,\dots\in\bigcap_{j=1}^\infty\Im_j,$$
		para todo $j=1,2,\dots$, $A_1,A_2,\dots\in\Im_j$, de modo que
		$$\bigcup_{i=1}^\infty A_i\in\Im_j\text{ y }\bigcup_{i=1}^\infty A_i\in\bigcap_{j=1}^\infty\Im_j.$$ 
	\end{proof}
	
	Con este teorema, se puede definir la $\sigma$-álgebra más pequeña\footnote{Es la más pequeña en el sentido de que es la que requiere menos elementos para satisfacer las condiciones necesarias para ser una $\sigma$-álgebra.} que contiene un subconjunto de $\Omega$. 	
	\begin{defi}[$\sigma$-álgebra generada]
		Tome $\Omega\neq\emptyset$ y $\mathcal{A}$ como una colección de subconjuntos de $\Omega$. Si $\mathcal{M}:=\{\Im:\Im\text{ es una }\sigma-\text{álgebra sobre }\Omega \text{ que contiene a } \mathcal{A}\},$
		$$\sigma(\mathcal{A}):=\bigcap_{\Im\in \mathcal{M}}\Im$$
		es la $\sigma-$álgebra más pequeña sobre $\Omega$ que contiene a $\mathcal{A}$. Esta $\sigma-$álgebra se conoce como $\sigma$-álgebra generada por $\mathcal{A}$.
	\end{defi}
	
	\begin{defi}[Espacio de medida]
		Tome $\Omega\neq\emptyset$ y sea $\Im$ una $\sigma$-álgebra sobre $\Omega$. La pareja $(\Omega,\Im)$ se llama espacio de medida.
	\end{defi}
	
	Al evento $\emptyset$ se le conoce como evento imposible; $\Omega$ es el evento seguro y $\{\omega\}$, con $\omega\in\Omega$ es un evento simple. Se dice que el evento $A$ ocurre después de llevar a cabo el experimento aleatorio si se obtiene un resultado en $A$, esto es, $A$ ocurre si el resultado es algún $\omega\in A$.
	\begin{enumerate}
		\item El evento $A\cup B$ ocurre si y sólo si $A$ ocurre, $B$ pasa, o ambos ocurren.
		\item El evento $A\cap B$ ocurre si y sólo si $A$ y $B$ ocurren a la vez.
		\item El evento $A^c$ ocurre si y sólo si $A$ no ocurre.
		\item El evento $A-B$ ocurre si y sólo si $A$ ocurre pero $B$ no ocurre.
	\end{enumerate}
	Si dos eventos no tienen eventos simples en común, se dirá que son eventos mutuamente excluyentes:
	\begin{defi}[Eventos mutuamente excluyentes]
		Dos eventos $A$ y $B$ se dicen mutuamente excluyentes si $A\cap B=\emptyset$.
	\end{defi}
	
	Antes de introducir la función de probabilidad, que medirá la posibilidad de que ocurra un evento de la $\sigma-$álgebra, es necesario definir por completez la frecuencia relativa, pues ella determina la posibilidad de que ocurra un evento al cabo de $n$ repeticiones del experimento aleatorio.
	\begin{defi}[Frecuencia relativa]
		Para cada evento $A$, el número $f_r(A):=\frac{n(A)}{n}$ se llama la frecuencia relativa de $A$, donde $n(A)$ indica el número de veces que ocurre $A$ en $n$ repeticiones del experimento aleatorio.
	\end{defi}
	Cuando $n\rightarrow\infty$, se puede hablar intuitivamente de la probabilidad de que ocurra el evento $A$, normalizada de $0$ a $1$. Por supuesto, es imposible realizar infinitas veces un experimento aleatorio para determinar la probabilidad de ocurrencia de todos los eventos de la $\sigma-$álgebra, por lo que se introduce de antemano la función de probabilidad, suponiendo que ella da cuenta del comportamiento de la frecuencia relativa cuando $n\rightarrow\infty$.
	
	\begin{defi}[Espacio de probabilidad]
		Tome $(\Omega,\Im)$ como un espacio de medida. Una función real $P$ sobre $\Im$ que satisface las siguientes condiciones:
		\begin{enumerate}
			\item $P(A)\geq0$ para todo $A\in\Im$ (no negativa),
			\item $P(\Omega)=1$ (normalizada) y,
			\item si $A_1,A_2,\dots$ son eventos mutuamente excluyentes en $\Im$, esto es, si
			$$A_i\cap A_j=\emptyset \text{ para todo }i\neq j, \text{ entonces}$$
			$$P\left(\bigcup_{i=1}^\infty A_i\right)=\sum_{i=1}^\infty P(A_i),$$
		\end{enumerate}
		se llama medida de probabilidad sobre $(\Omega,\Im)$. La tripleta $(\Omega,\Im,P)$ se llama espacio de probabilidad.
	\end{defi}
	El siguiente teorema caracteriza las propiedades más importantes de un espacio de probabilidad.
	\begin{theo}
		Si $(\Omega,\Im,P)$ es un espacio de probabilidad, entonces
		\begin{enumerate}
			\item $P(\emptyset)=0$.
			\item Si $A,B\in\Im$ y $A\cap B=\emptyset$, entonces $P(A\cup B)=P(A)+P(B).$
			\item Para todo $A\in\Im$, $P(A^c)=1-P(A).$
			\item Si $A\subseteq B$, entonces $P(A)\leq P(B)$ y $P(B-A)=P(B)-P(A).$ En particular, $P(A)\leq1$ para todo $A\in\Im$.
			\item Para todo $A,B\in\Im$, $P(A\cup B)=P(A)+P(B)-P(A\cap B)$.
		
		\item Tome $\{A_n\}_n\subseteq\Im$ como una sucesión creciente, esto es, $A_n\subseteq A_{n+1}, \forall n\in\mathbb{N}$; entonces
			$$P\left(\lim_{m\rightarrow\infty}A_n\right)=\lim_{n\rightarrow\infty}P(A_n), \text{ donde } \lim_{n\rightarrow\infty} A_n:=\bigcup_{i=1}^\infty A_i.$$
		\item Tome $\{A_n\}_n\subseteq\Im$ como una sucesión decreciente, esto es, $A_n\supseteq A_{n+1}, \forall n\in\mathbb{N}$; entonces
			$$P\left(\lim_{m\rightarrow\infty}A_n\right)=\lim_{n\rightarrow\infty}P(A_n), \text{ donde } \lim_{n\rightarrow\infty} A_n:=\bigcap_{i=1}^\infty A_i.$$
	\end{enumerate}
	\end{theo}
	\begin{proof}
		\begin{enumerate}
			\item $1=P(\Omega\cup\emptyset\cup\emptyset\cup\cdots)=P(\Omega)+P(\emptyset)+P(\emptyset)+\cdots=1+P(\emptyset)+\cdots \implies P(\emptyset)=0.$
			\item $P(A\cup B)=P(A\cup B\cup\emptyset\cup\emptyset\cup\cdots)=P(A)+P(B).$
			\item $P(A)+P(A^c)=P(A\cup A^c)=P(\Omega)=1\implies P(A^c)=1-P(A).$
			\item Si $A\subseteq B$, $B=A\cup(B-A)$, de modo que $P(B)=P(A)+P(B-A).$ Como $P\geq0$, $P(B)\geq P(A)$ y $P(B-A)=P(B)-P(A).$ Si $B=\Omega,$ $P(A)\leq1.$
			\item Use el hecho de que $A\cup B=[A-(A\cap B)]\cup[B-(A\cap B)]\cup[A\cap B]$.			
			\item Tome la sucesión $C_1=A_1,C_2=A_2-A_1,\dots,C_r=A_r-A_{r-1},\dots$. Es claro que
			$$\bigcup_{i=1}^\infty C_i=\bigcup_{i=1}^\infty A_i.$$
			Más aún, como $C_i\cap C_j=\emptyset\ \forall i\neq j,$ se sigue que
			$$P\left(\bigcup_{i=1}^\infty A_i\right)=P\left(\bigcup_{n=1}^\infty C_n\right)=\sum_{n=1}^\infty P(C_n)=\lim_{n\rightarrow\infty} \sum_{k=1}^n P(C_k)$$ $$=\lim_{n\rightarrow\infty}P\left(\bigcup_{k=1}^n C_k\right)=\lim_{n\rightarrow\infty}P\left(A_n\right).$$
			\item Tome la sucesión $\{B_n=A_n^c\}_n$ y aplique el resultado anterior.
		\end{enumerate}
	\end{proof}
Aplicando el teorema anterior de forma inductiva, para algunos eventos $A_1,A_2,\dots,A_n\in\Im$:
	\begin{align*}
	P(A_1\cup A_2\cup \cdots\cup A_n)=&\sum_{i=1}^n P(A_i)-\sum_{i_1<i_2}P(A_{i_{1}}\cap A_{i_{2}})+\cdots \\    
			& +(-1)^{r+1}\sum_{i_1<i_2<	\cdots<i_r}P(A_{i_1}\cap A_{i_2}\cap\cdots\cap A_{i_r})\\
			&+\cdots+(-1)^{n+1}P(A_1\cap A_2\cap \cdots\cap A_n).
	\end{align*}
	
	Por otro lado, tome $(\Omega,\Im,P)$ como un espacio de probabilidad con $\Omega$ finito o contable y $\Im =\mathbb{P}(\Omega)$. Tome $\emptyset\neq A \in\Im.$ Es claro que
	
	$$A=\bigcup_{\omega\in A} \{\omega\}, \text{ de modo que }$$
	
	$$P(A)=\sum_{\omega\in A} P(\omega), \text{ donde } P(\omega):=P(\{\omega\}).$$
	Así, $P$ queda completamente definido por $p_j:=P(\omega_j)$, donde $\omega_j\in\Omega.$ El vector $|\Omega|-$dimensional $p:=(p_1,p_2,\dots)$ satisface las siguientes condiciones:
	\begin{itemize}
		\item $p_j\geq0$ y
		\item $\sum_{j=1}^\infty p_j=1.$
	\end{itemize}
	Un vector que satisface las anteriores condiciones se llama \textbf{vector de probabilidad}.
	
	\subsection{Probabilidad condicional}
	Tome $B$ como un evento cuya opción de ocurrir debe ser medida bajo la suposición de que otro evento $A$ fue observado. Si el experimento se repite $n$ veces bajo las mismas circunstancias, entonces la frecuencia relativa de $B$ bajo la condición $A$ se define como
		$$f_r(B|A):=\frac{n(A\cap B)}{n(A)}=\frac{\frac{n(A\cap B)}{n}}{\frac{n(A)}{n}}=\frac{f_r(A\cap B)}{f_r(A)}, \text{ si }n(A)>0.$$
		Esto motiva la definición de probabilidad condicional, como el comportamiento de esta frecuencia relativa cuando $n\rightarrow\infty$.
		\begin{defi}[Probabilidad condicional]
			Tome $(\Omega, \Im, P)$ como un espacio de probabilidad. Si $A,B\in\Im$, con $P(A)>0$, entonces la probabilidad del evento $B$ bajo la condición $A$ se define como sigue
			$$P(B|A):=\frac{P(A\cap B)}{P(A).}$$
		\end{defi}
		El siguiente teorema provee algunas propiedades de la probabilidad condicional.
		\begin{theo}[Medida de probabilidad condicional]
			Tome $(\Omega, \Im, P)$ como un espacio de probabilidad y $A\in\Im$, con $P(A)>0$. Entonces:
			\begin{enumerate}
				\item $P(\cdot | A)$ es una medida de probabilidad sobre $\Omega$ centrada en $A$, esto es, $P(A|A)=1.$
				\item Si $A\cap B=\emptyset$, entonces $P(B|A)=0$.
				\item $P(B\cap C |A)=P(B|A\cap C)P(C|A)$ si $P(A\cap C)>0$.
				\item Si $A_1,A_2,\dots,A_n\in\Im$, con $P(A_1\cap A_2\cap\cdots\cap A_{n-1})>0$, entonces
				\begin{align*}
				P(A_1\cap A_2\cap\cdots\cap A_n)=P(A_1)P(A_2|A_1)P(A_3|A_1\cap A_2)\cdots P(A_n|A_1\cap A_2\cap \cdots\cap A_{n-1}).
				\end{align*}
			\end{enumerate}
		\end{theo}		
		
		\begin{proof}
			\begin{enumerate}
				\item Las tres propiedades de una medida de probabilidad deben ser verificadas.
				\begin{enumerate}
					\item Claramente, $P(B|A)\geq0$ para todo $B\in\Im$.
					\item $P(\Omega | A)=\frac{P(\Omega\cap A)}{P(A)}=\frac{P(A)}{P(A)}=1.$ También se tiene que $P(A|A)=1.$
					\item Tome $A_1,A_2,\dots\in\Im$ una sucesión de conjuntos disyuntos. Entonces
					\begin{align*}
					P\left(\bigcup_{i=1}^\infty A_i | A\right)=\frac{P\left(A\cap \bigcup_{i=1}^\infty A_i\right)}{P(A)}=\frac{P\left(\bigcup_{i=1}^\infty A\cap A_i\right)}{P(A)}\\
					=\sum_{i=1}^\infty \frac{P(A\cap A_i)}{P(A)}=\sum_{i=1}^\infty P(A_i | A).
					\end{align*}
				\end{enumerate}
				\item Si $A\cap B=\emptyset$, $P(B|A)=\frac{P(A\cap B)}{P(A)}=\frac{P(\emptyset)}{P(A)}=0.$
				\item $P(B\cap C|A)=\frac{P(A\cap B\cap C)}{P(A)}=\frac{P(B\cap C\cap A)}{P(A\cap C)}\frac{P(C\cap A)}{P(A)}=P(B|A\cap C)P(C| A).$
				\item $P(A_1\cap\cdots\cap A_n)=\frac{P(A_1\cap\cdots\cap A_n)}{P(A_1\cap\cdots\cap A_{n-1})}\frac{P(A_1\cap\cdots\cap A_{n-1})}{P(A_1\cap\cdots\cap A_{n-2})}\cdots\frac{P(A_1\cap A_2)}{P(A_1)}P(A_1)=P(A_1)P(A_2|A_1)P(A_3|A_1\cap A_2)\cdots P(A_n|A_1\cap A_2\cap \cdots\cap A_{n-1}).$
			\end{enumerate}
			\end{proof}	
			El siguiente teorema permitirá probar la regla de Bayes.
			
		\begin{theo}[Teorema de probabilidad total]
			Tome $A_1,A_2,\dots$ como una partición finita o contable de $\Omega$, esto es, $A_i\cap A_j=\emptyset, \forall i\neq j$ y $\bigcup_{i=1}^\infty A_i=\Omega,$ tal que $P(A_i)>0$, para todo $A_i\in\Im$. Entonces, para todo $B\in\Im$:
			$$P(B)=\sum_{i} P(B|A_i)P(A_i).$$
		\end{theo}
		\begin{proof}
			Observe que
			$$B=B\cap\Omega=B\cap\left( \bigcup_{i=1}^\infty A_i \right)=\bigcup_{i=1}^\infty B\cap A_i,$$
			de modo que
			$$P(B)=P\left(\bigcup_{i=1}^\infty B\cap A_i\right)=\sum_{i=1}^\infty P(B\cap A_i)=\sum_{i=1}^\infty P(B| A_i)P(A_i).$$
		\end{proof}
		Matemáticamente, este teorema se puede interpretar como que la probabilidad de que ocurra $B$ se puede medir en términos de una partición de $\Omega$ en el sentido de que $B$, como subconjunto de $\Omega$ puede ocurrir cuando ocurran algunos elementos de la partición, los cuales tendrán mayor peso en el término $P(B|A_i)$ de la suma.
		
		Como corolario del teorema anterior, se obtiene la \textbf{regla de Bayes}, que constituye la base para la \textbf{teoría Bayesiana}.
		\begin{corol}[Regla de Bayes]
			Tome $A_1,A_2,\dots$ como una partición finita o contable de $\Omega$ con $P(A_i)>0$, para todo $i$; entonces, para todo $B\in\Im$ con $P(B)>0:$
			$$P(A_i | B)=\frac{P(A_i)P(B| A_i)}{\sum_j P(B|A_j)P(A_j)}, \forall i.$$
		\end{corol}
		\begin{proof}
		$$P(A_i|B)=\frac{P(A_i\cap B)}{P(B)}=\frac{P(A_i)P(B|A_i)}{P(B)}=\frac{P(A_i)P(B| A_i)}{\sum_j P(B|A_j)P(A_j)}.$$
		Con la partición $A_1=A, A_2=A^c$ se obtiene la forma usual de la regla de Bayes.
	\end{proof}
	
	A continuación se definen las distribuciones a priori y a posteriori, que hacen referencia a la probabilidad de que ocurran ciertos eventos de una partición de $\Omega$ antes de que ocurra un evento $B$ y al cabo de que este evento $B$ ocurra.
	\begin{defi}[Distribuciones a priori y a posteriori]
		Tome $A_1,A_2,\dots$ como una partición finita o contable de $\Omega$, con $P(A_i)>0$, para todo $i$. Si $P(B)>0$, con $B\in\Im$, entonces $\{P(A_n)\}_n$ se llama distribución a priori (antes de que $B$ ocurra), y $\{P(A_n|B)\}_n$ se llama distribución a posteriori (después de que $B$ ocurra).
	\end{defi}
	Algunas veces, la ocurrencia de un evento $B$ no afecta la probabilidad de un evento $A$, es decir,
	$$P(A|B)=P(A).$$
	En este caso, se dice que el evento $A$ es independiente del evento $B$. Esto motiva la siguiente definición.
	\begin{defi}[Eventos independientes]
		Dos eventos $A$ y $B$ se dicen independientes si y sólo si
		
		$$P(A\cap B)=P(A)P(B).$$
		Si esta condición no se tiene, se dice que los eventos son dependientes.
	\end{defi}
	
	En algunos casos, es necesario analizar la independencia de dos o más eventos. Para ello, se dan las siguientes definiciones.
	\begin{defi}[Familia independiente]
		Una familia de eventos $\{A_i: i\in I\}$ se dice independiente si
		$$P\left(\bigcap_{i\in J} A_i\right)=\prod_{i\in J}P(A_i),$$
		para cualquier subconjunto no vacío $J\subseteq I$.
	\end{defi}
	\begin{defi}[Eventos independientes par a par]
		Una familia de eventos $\{A_i: i\in I\}$ se dice par a par independiente si
		$$P(A_i\cap A_j)=P(A_i)P(A_j), \text{ para todo }i\neq j.$$
	\end{defi}
	
	\subsection{Variables aleatorias}
	En un experimento aleatorio, generalmente hay mayor interés en determinar ciertos valores numéricos asociados a los resultados del experimento aleatorio, que al resultado mismo del experimento aleatorio. Con esto en mente se define la variable aleatoria.

	\begin{defi}[Variable aleatoria]
	Tome $(\Omega, \Im, P)$ como un espacio de probabilidad. Una variable aleatoria es un mapa $X:\Omega\rightarrow \mathbb{R}$ tal que, para todo $A\in\mathbb{B}$, $X^{-1}(A)\in\Im$, donde $\mathbb{B}$ es la $\sigma$-álgebra de Borel sobre $\mathbb{R}$ ($\sigma$-álgebra más pequeña que contiene todos los intervalos de la forma $(-\infty, a]$).
	
	El conjunto de posibles valores de $X$ es $\mathbb{S}:=\{x\in\mathbb{R}:\exists \omega\in\Omega\text{ tal que }X(\omega)=x\}$, conocido como \textbf{soporte de la variable aleatoria }$X$.
	\end{defi}
	Si $X$ es una variable aleatoria definida sobre un  espacio de probabilidad $(\Omega, \Im, P)$, se introduce la notación
	$$\{X\in B\}:=\{\omega\in\Omega: X(\omega)\in B\}, \text{ con }B\in\mathbb{B}.$$
	\begin{defi}[Variable aleatoria discreta]
		Una variable aleatoria $X$ se dice discreta cuando el soporte $\mathbb{S}$ de $X$ es un subconjunto finito o contable de $\mathbb{R}$. Para $x\in\mathbb{S}$, la función $f(x)=P(X=x):=P($ se llama función de densidad de probabilidad (pdf para abreviar).
	\end{defi}
	
	\begin{defi}[Variable aleatoria continua]
		Una variable aleatoria $X$ se dice continua si el soporte $\mathbb{S}$ de $X$ es la unión de uno o más intervalos y si existe una función no negativa y real $f(x)$ tal que $P(X\leq x)=\int_{-\infty}^x f(t)\mathrm{d}t.$ La función $f(x)$ se llama función de densidad de probabilidad (pdf).
	\end{defi}
	
	Algunas propiedades de la \textit{pdf} discreta son las siguientes:
	\begin{enumerate}
		\item $f(x)\geq0, \forall x\in\mathbb{S}$ y $f(x)=0,\forall x\not\in \mathbb{S}$.
		\item $\sum_{x\in\mathbb{S}}f(x)=1$.
		\item $P(X\in B)=\sum_{x\in B} f(x).$
	\end{enumerate}
	Análogamente, para la \textit{pdf} continua:
	\begin{enumerate}
		\item $f(x)\geq0, \forall x\in\mathbb{S}$ y $f(x)=0,\forall x\not\in \mathbb{S}$.
		\item $\int_{\mathbb{S}}f(x)\mathrm{d}x=1$.
		\item $P(X\in B)=\int_{B}f(x)\mathrm{d}x.$
	\end{enumerate}
	\begin{defi}[Función de distribución acumulativa]
		La función de distribución acumulativa (CDF, para abreviar) de una variable aleatoria se define como la función $F(x)=P(X\leq x)$.
	\end{defi}
	El siguiente teorema resume algunas propiedades importantes de una \textit{CDF}. 
	
	\begin{theo}
		Si $X$ es una variable aleatoria, con \textit{CDF} $F(x)$, entonces:
		\begin{enumerate}
			\item $\lim_{x\rightarrow-\infty}F(x)=0$ y $\lim_{x\rightarrow\infty} F(x)=1.$
			\item $F(x)$ es no decreciente; esto es, $F(x)\leq F(y)$, siempre que $x\leq y$.
			\item $F(x)$ es continua por derecha.
			\item $P(a< X \leq b)=F(b)-F(a).$
		\end{enumerate}
	\end{theo}
	
	\begin{proof}

	\begin{enumerate}
		
		\item \textbf{Cuando $\mathbf{x\rightarrow-\infty}$}, para todo $n\in\mathbb{N}$, se satisface que
		$$\{X\leq -n\} \supseteq \{X\leq -(n+1)\}, \text{ y en adición,}$$
		$$\emptyset = \bigcap_{n=1}^\infty \{ X\leq -n\}, \text{ con lo que}$$
		$$\lim_{x\rightarrow-\infty}F(x)=\lim_{n\rightarrow \infty}F(-n)=P(\emptyset)=0.$$
		
		\textbf{Cuando $\mathbf{x\rightarrow\infty}$}, para todo $n\in\mathbb{N}$, se satisface que
		$$\{X\leq n\}\subseteq\{ X\leq n+1\}, \text{ y,}$$
		$$\Omega=\bigcup_{n=1}^\infty \{ X\leq n\}, \text{ de modo que}$$
		$$\lim_{x\rightarrow\infty} F(x)=\lim_{n\rightarrow\infty}=P(\Omega)=1.$$
		
		\item Si $x\leq y$, entonces
		$$\{X\leq x\}\subseteq \{X\leq y\}, \textit{ con lo que}$$
		$$F(x)=P(X\leq x)\leq P(X\leq y)=F(y).$$
		
		\item Tome $x\in\mathbb{R}$ fijo. Suponga que $\{x_n\}_{n\in\mathbb{N}}$ es una sucesión decreciente de números reales, cuyo límite va a $x$. Se puede ver que
		$$\{X\leq x_1\}\supseteq\{X\leq x_2\}\supseteq\cdots, \text{ y}$$
		$$\bigcap_{n=1}^\infty \{X\leq x_n\}=\{ X\leq x\}, \text{ de modo que}$$
		$$\lim_{y\rightarrow x^+} F(y)=\lim_{n\rightarrow \infty} F(x_n)=\lim_{n\rightarrow\infty} P(X\leq x_n)=P(X\leq x)=F(x).$$
		
		\item Dado que $\Omega=\{X\leq a\}\cup\{a<X\leq b\}\cup \{X>b\},$
		$$1=P(X\leq a)+P(a<X\leq b)+P(X>b),$$
		$$\therefore P(a<X\leq b)=P(X\leq b)-P(X\leq a)=F(b)-F(a).$$
	\end{enumerate}
		
	\end{proof}
	
	Los teoremas posteriores indican cómo determinar la \textit{pdf} a partir de la \textit{CDF} de una variable aleatoria.
	\begin{theo}
		Si $X$ es una variable aleatoria discreta con \textit{CDF} $F(x)$ y soporte $\mathbb{S}=\{x_0,x_1,\dots\}$, con $x_0<x_1<\cdots$, entonces, para $x_k\in\mathbb{S}$,
		$$f(x_k)=F(x_k)-F(x_{k-1}).$$
	\end{theo}
	\begin{proof}
		Para $k\geq 1$,
		$$f(x_k)=P(X=x_k)=P(x_{k-1}<X\leq x_k)=F(x_k)-F(x_{k-1}).$$
	\end{proof}
	\begin{theo}
		Para una variable aleatoria continua, $f(x)=\frac{\mathrm{d}F}{\mathrm{d}x}, \forall x\in\mathbb{R}.$
	\end{theo}
	\begin{proof}
		La prueba se sigue directamente del teorema fundamental del cálculo.
	\end{proof}
	\subsection{Vectores aleatorios}
	En la mayoría de los análisis estadísticos, más de una variable debe ser analizada al cabo de un experimento aleatorio. Cada observación se puede representar como un vector de observaciones, conocido como vector aleatorio.
	
	\begin{defi}[Vector aleatorio]
		Un vector aleatorio $\vec{X}=(X_1,X_2,\dots,X_k)$ es un vector $k-$dimensional, donde $X_1,\dots,X_k$ son variables aleatorias. Un vector aleatorio se dice discreto cuando cada una de las variables aleatorias que lo conforman son discretas, y continuo cuando son continuas.
	\end{defi}
	
	\begin{defi}[Variable aleatoria bivariada]
		Un vector aleatorio bidimensional $\vec{X}=(X_1,X_2)$ se llama variable aleatoria bivariada.
	\end{defi}
	
	De modo similar al caso de las variables aleatorias, los vectores aleatorios tienen \textit{pdf}, un soporte y una \textit{CDF}. El soporte de un vector aleatorio $k-$dimensional es el conjunto de valores que puede tomar, denotado por $\mathbb{S}_{\vec{X}}\subseteq\mathbb{R}^k.$
	
	\begin{defi}[Función de densidad de probabilidad adjunta discreta]
		Tome $\vec{X}$ como un vector aleatorio discreto $k-$dimensional. La \textit{pdf} adjunta de $\vec{X}$ se define como
		$$f(\vec{x}):=f(x_1,x_2,\dots,x_k)=P(X_1=x_1,X_2=x_2,\dots, X_k=x_k)$$
		para $\vec{x}=(x_1,x_2,\dots,x_k)\in\mathbb{S}_{\vec{X}}.$
	\end{defi}
	La \textit{pdf} adjunta discreta tiene las siguientes propiedades:
	\begin{enumerate}
		\item $0\leq f(x_1,x_2,\dots,x_k)\leq1, \forall \vec{x}\in\mathbb{S}_{\vec{X}}.$
		\item $\sum_{\vec{x}\in\mathbb{S}_{\vec{X}}}f(\vec{x})=1.$
		\item Para cualquier subconjunto $B\subseteq \mathbb{S}_{\vec{X}},\ P(\vec{X}\in B)=\sum_{\{\vec{x}\in\mathbb{S}_{\vec{X}}:\vec{x}\in B\}}f(\vec{x}).$
	\end{enumerate}
	
	\begin{defi}[Función de densidad de probabilidad adjunta continua]
		Tome $\vec{X}$ como un vector aleatorio $k$-dimensional continuo. La \textit{pdf} continua de $\vec{X}$ se define como cualquier función no negativa $f(\vec{x})$ que satisfaga las siguientes propiedades:
		\begin{enumerate}
			\item $f(x_1,\dots,x_k)>0, \forall \vec{x}\in\mathbb{S}_{\vec{X}}.$
			\item $\int_{\mathbb{S}_{\vec{X}}} f(x_1,\dots,x_k)\mathrm{d}x_1\cdots\mathrm{d}x_k=1.$
			\item Para cualquier subconjunto $B\subset\mathbb{S}_{\vec{X}},\ P(\vec{X}\in B)=\int_{B}f(x_1,\dots,x_k)\mathrm{d}x_1\cdots\mathrm{d}x_k.$
		\end{enumerate}
	\end{defi}
	Un vector aleatorio también tiene \textit{CDF}, conocida en este caso como función de distribución acumulativa adjunta o \textit{CDF} adjunta.
	\begin{defi}[Función de distribución acumulativa adjunta]
		Tome $\vec{X}=(X_1,X_2,\dots,X_k)$ como un vector aleatorio $k-$dimensional. La \textit{CDF} adjunta de $\vec{X}$ se define como
		$$F(x_1,x_2,\dots,x_k)=P(X_1\leq x_1, X_2\leq x_2,\dots, X_k\leq x_k), \ \forall (x_1,\dots,x_k)\in\mathbb{R}^k.$$
	\end{defi}
	Las componentes de un vector aleatorio $\vec{X}$ son variables aleatorias, por lo que las \textit{pdf} de cada variable aleatoria $X_i$ de $\vec{X}$ se pueden derivar de la \textit{pdf} adjunta de $\vec{X}$.
	\begin{defi}[Función de densidad de probabilidad marginal]
		Tome $\vec{X}=(X_1,\dots,X_k)$ como un vector aleatorio $k-$dimensional. La función de densidad de probabilidad marginal de la variable aleatoria $X_i$ es
		
		\begin{center}
		\begin{tabular}{cc}
		$f_i(x_i)=\underbrace{\sum_{x_1\in\mathbb{S}_{X}}\cdots \sum_{x_k\in\mathbb{S}_{X_k}}}_{\text{quitando la suma sobre }x_i} f(x_1,\dots,x_k)$ & cuando $\vec{X}$ es discreta y\\
		$f_i(x_i)=\underbrace{\int_{x_1\in\mathbb{S}_{X_1}}\cdots \int_{x_k\in\mathbb{S}_{X_k}}}_{\text{quitando la integral sobre }x_i} f(x_1,\dots,x_k)\prod_{n\neq i}\mathrm{d}x_n$ & cuando $\vec{X}$ es continua.
		\end{tabular}
		\end{center}
	\end{defi}
	También se puede definir la distribución condicional dada una variable aleatoria de forma similar al caso de la probabilidad condicional.
	\begin{defi}[Función de densidad de probabilidad condicional]
		Tome $\vec{X}(X_1,\dots,X_k)$ como un vector aleatorio $k-$dimensional. Para un valor fijo de $x_i$, donde $f_i(x_i)>0$, la función de densidad de probabilidad condicional para $\vec{Y}|X_i$; donde $\vec{Y}$ es un vector aleatorio $(k-1)-$dimensional con todas las variables aleatorias de $\vec{X}$, a excepción de $X_i$; es
		$$f(\vec{y}|x_i)=\frac{f(x_1,\dots,x_k)}{f_i(x_i)},$$
		donde $\vec{y}\in\mathbb{S}_{\vec{Y}}.$
	\end{defi}